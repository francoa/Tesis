%%%%%%%%%%%%%%%%%%%%%%%%%%%%%%%%%%%%%%%%%
% Masters/Doctoral Thesis 
% LaTeX Template
% Version 1.43 (17/5/14)
%
% This template has been downloaded from:
% http://www.LaTeXTemplates.com
%
% Original authors:
% Steven Gunn 
% http://users.ecs.soton.ac.uk/srg/softwaretools/document/templates/
% and
% Sunil Patel
% http://www.sunilpatel.co.uk/thesis-template/
%
% License:
% CC BY-NC-SA 3.0 (http://creativecommons.org/licenses/by-nc-sa/3.0/)
%
% Note:
% Make sure to edit document variables in the Thesis.cls file
%
%%%%%%%%%%%%%%%%%%%%%%%%%%%%%%%%%%%%%%%%%

%----------------------------------------------------------------------------------------
%	PACKAGES AND OTHER DOCUMENT CONFIGURATIONS
%----------------------------------------------------------------------------------------

\documentclass[10pt, oneside]{Thesis} % The default font size and one-sided printing (no margin offsets)

\graphicspath{{Figures/}} % Specifies the directory where pictures are stored

\usepackage[square, comma, sort&compress]{natbib} % Use the natbib reference package - read up on this to edit the reference style; if you want text (e.g. Smith et al., 2012) for the in-text references (instead of numbers), remove 'numbers' 
\usepackage[spanish]{babel}
\usepackage{subfig}
\usepackage{titlesec}
\hypersetup{urlcolor=blue, colorlinks=true} % Colors hyperlinks in blue - change to black if annoying
\title{\ttitle} % Defines the thesis title - don't touch this

\begin{document}

\frontmatter % Use roman page numbering style (i, ii, iii, iv...) for the pre-content pages

\setstretch{1.3} % Line spacing of 1.3

% Define the page headers using the FancyHdr package and set up for one-sided printing
\fancyhead{} % Clears all page headers and footers
\rhead{\thepage} % Sets the right side header to show the page number
\lhead{} % Clears the left side page header

\pagestyle{fancy} % Finally, use the "fancy" page style to implement the FancyHdr headers

\newcommand{\HRule}{\rule{\linewidth}{0.5mm}} % New command to make the lines in the title page

% PDF meta-data
\hypersetup{pdftitle={\ttitle}}
\hypersetup{pdfsubject=\subjectname}
\hypersetup{pdfauthor=\authornames}
\hypersetup{pdfkeywords=\keywordnames}


\titleformat{\chapter}[display]
  {\normalfont\huge\bfseries\color{black}}
  {\chaptertitlename\ \thechapter}{20pt}{}

%----------------------------------------------------------------------------------------
%	TITLE PAGE
%----------------------------------------------------------------------------------------

\begin{titlepage}
\begin{center}

\textsc{\LARGE \univname}\\[1.5cm] % University name
\textsc{\Large Proyecto Final de Estudios}\\[0.5cm] % Thesis type

\HRule \\[0.4cm] % Horizontal line
{\huge \bfseries \ttitle}\\[0.4cm] % Thesis title
\HRule \\[1.5cm] % Horizontal line
 
\begin{minipage}{0.4\textwidth}
\begin{flushleft} \large
\emph{Autores:}\\
{\authornames} % Author name - remove the \href bracket to remove the link
\end{flushleft}
\end{minipage}
\begin{minipage}{0.4\textwidth}
\begin{flushright} \large
\emph{Supervisores:} \\
{\supname} % Supervisor name - remove the \href bracket to remove the link  
\end{flushright}
\end{minipage}\\[3cm]
 
\large \textit{Tesis presentada en cumplimiento de los requerimientos\\ para el grado de \degreename}\\[0.3cm] % University requirement text
\textit{en}\\[0.4cm]
\facname\\\univname\\[2cm] % Research group name and department name
 
{\large \today}\\[4cm] % Date
%\includegraphics{Logo} % University/department logo - uncomment to place it
 
\vfill
\end{center}

\end{titlepage}

%----------------------------------------------------------------------------------------
%	DECLARATION PAGE
%	Your institution may give you a different text to place here
%----------------------------------------------------------------------------------------

\Declaration{

\addtocontents{toc}{\vspace{1em}} % Add a gap in the Contents, for aesthetics

Nosotros, \authorx \hspace{2mm}y \authory, declaramos que esta tesis titulada, '\ttitle' y el trabajo presentado son nuestros. Confirmamos que:

\begin{itemize} 
\item[\tiny{$\blacksquare$}] \Large{PONER AQUI LOS ITEMS DE LA DECLARACION DE AUTORIA}
\\
\end{itemize}
 
Firmas:\\
\rule[1em]{36em}{0.5pt} % This prints a line for the signature
 
Fecha:\\
\rule[1em]{36em}{0.5pt} % This prints a line to write the date
}

\clearpage % Start a new page

%----------------------------------------------------------------------------------------
%	QUOTATION PAGE
%----------------------------------------------------------------------------------------

\pagestyle{empty} % No headers or footers for the following pages

%\null\vfill % Add some space to move the quote down the page a bit

%\textit{``\Large{UNA FRASE RE COPADA"}

%\begin{flushright}
%Dave Barry
%\end{flushright}

%\vfill\vfill\vfill\vfill\vfill\vfill\null % Add some space at the bottom to position the quote just right

%\clearpage % Start a new page

%----------------------------------------------------------------------------------------
%	ABSTRACT PAGE
%----------------------------------------------------------------------------------------

\addtotoc{Resumen} % Add the "Abstract" page entry to the Contents

\abstract{\addtocontents{toc}{\vspace{1em}} % Add a gap in the Contents, for aesthetics

\Large{ESCRIBIR ABSTRACT}\ldots
}

\clearpage % Start a new page

%----------------------------------------------------------------------------------------
%	ACKNOWLEDGEMENTS
%----------------------------------------------------------------------------------------

\setstretch{1.3} % Reset the line-spacing to 1.3 for body text (if it has changed)

\acknowledgements{\addtocontents{toc}{\vspace{1em}} % Add a gap in the Contents, for aesthetics

\Large{AGRADECIMIENTOS ACA, don't forget to include your project advisor\ldots}
}
\clearpage % Start a new page

%----------------------------------------------------------------------------------------
%	LIST OF CONTENTS/FIGURES/TABLES PAGES
%----------------------------------------------------------------------------------------

\pagestyle{fancy} % The page style headers have been "empty" all this time, now use the "fancy" headers as defined before to bring them back

\lhead{\emph{Contents}} % Set the left side page header to "Contents"
\tableofcontents % Write out the Table of Contents

\lhead{\emph{List of Figures}} % Set the left side page header to "List of Figures"
\listoffigures % Write out the List of Figures

\lhead{\emph{List of Tables}} % Set the left side page header to "List of Tables"
\listoftables % Write out the List of Tables

%----------------------------------------------------------------------------------------
%	ABBREVIATIONS
%----------------------------------------------------------------------------------------

\clearpage % Start a new page

\setstretch{1.5} % Set the line spacing to 1.5, this makes the following tables easier to read

\lhead{\emph{Abbreviations}} % Set the left side page header to "Abbreviations"
\listofsymbols{ll} % Include a list of Abbreviations (a table of two columns)
{
\textbf{BMG} & \textbf{B}ulk \textbf{M}etalic \textbf{G}lass \\
\textbf{CSM} & \textbf{C}ooperative \textbf{S}hear \textbf{M}odel \\
\textbf{EAM} & \textbf{E}mbedded \textbf{A}tom \textbf{M}ethod \\
\textbf{SB} & \textbf{S}hear \textbf{B}and \\
\textbf{STZ} & \textbf{S}hear \textbf{T}ransformation \textbf{Z}one \\
\textbf{SVF} & \textbf{S}olid \textbf{V}olume \textbf{F}raction \\
\textbf{MD} & \textbf{M}olecular \textbf{D}ynamics \\
%\textbf{Acronym} & \textbf{W}hat (it) \textbf{S}tands \textbf{F}or \\
}

% STZ : zona de transformación de tensión cortante
% SB : bandas de corte
% shear strain : deformación cortante
% nanovidrio, nanopartículas, nanoporos
% BMG : vidrios metálicos masivos
% uniaxial : uniaxial
% compressive strain : deformación por compresión

%----------------------------------------------------------------------------------------
%	PHYSICAL CONSTANTS/OTHER DEFINITIONS
%----------------------------------------------------------------------------------------

\clearpage % Start a new page

\lhead{\emph{Physical Constants}} % Set the left side page header to "Physical Constants"

\listofconstants{lrcl} % Include a list of Physical Constants (a four column table)
{
Constante de Boltzmann & $k_{B}$ & $=$ & $1.380\ 65\ \times10^{-23}\ \mbox{J}\mbox{K}^{-\mbox{1}}$ \\
% Constant Name & Symbol & = & Constant Value (with units) \\
}

%----------------------------------------------------------------------------------------
%	SYMBOLS
%----------------------------------------------------------------------------------------

\clearpage % Start a new page

\lhead{\emph{Symbols}} % Set the left side page header to "Symbols"

\listofnomenclature{lll} % Include a list of Symbols (a three column table)
{

$E$ & módulo de Young & GPa \\
$G$ & módulo de corte & GPa \\
$R$ & coeficiente de correlación &  \\
$T$ & temperatura & K \\
$T_{g}$ & temperatura de transición vítrea & K \\
% Symbol & Name & Unit \\

& & \\ % Gap to separate the Roman symbols from the Greek

$\epsilon$ & deformación ingenieril &  \\
$\sigma$ & tensión & GPa \\
$\sigma_{y}$ & tensión de fluencia & GPa \\
$\sigma_{VM}$ & tensión de von Mieses & GPa \\
% Symbol & Name & Unit \\
}

%----------------------------------------------------------------------------------------
%	DEDICATION
%----------------------------------------------------------------------------------------

\setstretch{1.3} % Return the line spacing back to 1.3

\pagestyle{empty} % Page style needs to be empty for this page

\dedicatory{\Large{DEDICATORIA}\ldots} % Dedication text

\addtocontents{toc}{\vspace{2em}} % Add a gap in the Contents, for aesthetics

%----------------------------------------------------------------------------------------
%	THESIS CONTENT - CHAPTERS
%----------------------------------------------------------------------------------------

\mainmatter % Begin numeric (1,2,3...) page numbering

\pagestyle{fancy} % Return the page headers back to the "fancy" style

% Include the chapters of the thesis as separate files from the Chapters folder
% Uncomment the lines as you write the chapters

% Chapter Template

\chapter{INTRODUCCION} % Main chapter title

\label{C1} % Change X to a consecutive number; for referencing this chapter elsewhere, use \ref{ChapterX}

\lhead{Capítulo 1. \emph{INTRODUCCION}} % Change X to a consecutive number; this is for the header on each page - perhaps a shortened title

%----------------------------------------------------------------------------------------
%	SECTION 1
%----------------------------------------------------------------------------------------

\section{Descripción del problema y motivación. Tamaño (nano) de estudio}
\label{S1_1}

\Large{Una referencia, para que no joda mas: } \citep{albe13}.\\

%----------------------------------------------------------------------------------------
%	SECTION 2
%----------------------------------------------------------------------------------------

\section{Antecedentes (Citar muchos papers)}
\label{S1_2}

%-----------------------------------
%	SUBSECTION
%-----------------------------------

\subsection{Teóricos}
\label{S1_2_1}

%-----------------------------------
%	SUBSECTION
%-----------------------------------

\subsection{Experimentales}
\label{S1_2_2}

%-----------------------------------
%	SUBSECTION
%-----------------------------------

\subsection{NUEVOS}
\label{S1_2_3}

%-----------------------------------
%	SUBSECTION
%-----------------------------------

\subsection{Sobre FEM -> continuo}
\label{S1_2_4}

%-----------------------------------
%	SUBSECTION
%-----------------------------------

\subsection{Sobre MD -> atómico}
\label{S1_2_5}

%----------------------------------------------------------------------------------------
%	SECTION 3
%----------------------------------------------------------------------------------------

\section{Objetivos}
\label{S1_3}

%----------------------------------------------------------------------------------------
%	SECTION 4
%----------------------------------------------------------------------------------------

\section{Contenido del PFE}
\label{S1_4}

d

d

d

d

d
% Chapter Template

\chapter{ASPECTOS TEORICOS BASICOS} % Main chapter title

\label{C2} % Change X to a consecutive number; for referencing this chapter elsewhere, use \ref{ChapterX}

\lhead{Capítulo 2. \emph{ASPECTOS TEORICOS BASICOS}} % Change X to a consecutive number; this is for the header on each page - perhaps a shortened title

%----------------------------------------------------------------------------------------
%	SECTION 1
%----------------------------------------------------------------------------------------

\section{Simulación, Teoría y Experimentos}
\label{S2_1}

Las simulaciones en computadora han abierto la posibilidad a estudiar sistemas, y verificar modelos bajo condiciones que son imposibles (o muy difíciles) de conseguir experimentalmente. También han permitido incrementar la complejidad de los modelos usados y realizar comparaciones directas con fenómenos naturales, una vez que el modelo ha sido validado.

Con la llegada de computadoras más potentes, aparece una nueva opción a considerar entre teoría y experimentos: los experimentos simulados. El modelo sobre el cual se realiza esta simulación sigue siendo teórico, pero todos los cálculos son llevados a cabo por la computadora. Es muy importante por supuesto, definir las condiciones de simulación de tal manera que represente una situación físicamente posible, y no obtener resultados absurdos.

Tanto el modelo como las condiciones de simulación y los elementos que componen el experimento simulado se encuentran en el marco de algún paquete de software que mediante algoritmos específicos resuelven el problema y nos entregan ciertos resultados. En este caso particular se hace uso de simulaciones de Dinámica Molecular (apropiadas para estudiar fenómenos en escala nano) mediante el paquete LAMMPS \citep{plimpton95}, que es de código abierto y gratuito.

%----------------------------------------------------------------------------------------
%	SECTION 2
%----------------------------------------------------------------------------------------

\section{Introducción a MD}
\label{S2_2}

Las simulaciones de dinámica molecular (MD) son usadas frecuentemente para estudiar propiedades a escala nano \citep{allen87}. Las simulaciones de MD son una técnica muy poderosa que permite resolver, usando mecánica clásica, problemas con muchos cuerpos, dada una interacción entre átomos. Una ventaja de MD es que la deformación, el esfuerzo, la temperatura, la velocidad, etc. son todas conocidas en detalle \citep{allen87}. Con esta información un gran número de fenómenos pueden estudiarse, como cambios de fase, transferencia de calor, creación y movimiento de dislocaciones, defectos, etc.. MD es una herramienta muy versátil para el estudio de las propiedades de materiales, y hasta ha sido utilizada para predecir el comportamiento mecánico de materiales previamente a los experimentos, como en el caso de maclas de aluminio \citep{chen03}. Las simulaciones de MD reproducen el movimiento atómico y por lo tanto emplean pasos de tiempo de 1 fs \citep{allen87}.

Las simulaciones de MD integran en el tiempo las ecuaciones de movimiento de Newton para un grupo de $N$ átomos, dadas por:

\begin{equation}
\mathbf{F_{i}} = m_{i}\mathbf{a_{i}}
\end{equation}

donde $m_{i}$ es la masa de cada átomo y $\mathbf{a_{i}}$ es su aceleración, dada por $\frac{d^{2}\mathbf{r_{i}}}{dt^{2}}$. Esto quiere decir que el resultado final de una simulación de MD es uno sólo y está determinado por las posiciones y velocidades iniciales (los errores de redondeo propios del cálculo numérico hacen que las soluciones diverjan de todas maneras). Vale la pena aclarar que si bien en sistemas pequeños, dos condiciones iniciales diferentes pueden llevar a soluciones muy distintas. En sistemas con un gran número de átomos las soluciones deberían ser estadísticamente comparables.

El estado termodinámico del sistema está definido por ciertos parámetros como la temperatura, la presión, el número de partículas, etc. El estado microscópico del sistema queda definido por las posiciones $p_{i}$ y los momentos atómicos $q_{i}$ y lo llamamos \textbf{espacio de fase}. Las posiciones y los momentos son considerados coordenadas de un espacio $6N$-dimensional ($\Omega$), en el que un punto cualquiera se denomina $P$ y describe el estado del sistema. 

Todas las configuraciones posibles que tienen diferente estado microscópico pero tienen idéntico estado macroscópico o termodinámico se denomina \textbf{ensamble estadístico}. En una simulación de MD, se aplican restricciones o condiciones externas, que determinan el tipo de ensamble, es decir, las sucesivos valores de velocidad y posición que obtenemos en cada paso de la simulación, son configuraciones diferentes del mismo ensamble.

Las propiedades termodinámicas se calculan como promedios de ciertas funciones que aplican sobre la configuración del sistema (un \textbf{observable} $\mathbf{\mathcal{A}}(p_{i},q_{i})$) sobre todas las configuraciones posibles.

\begin{equation}
\langle \mathbf{\mathcal{A}} \rangle _{ens}
\end{equation}

Esto resulta extremadamente difícil dado el número enorme de configuraciones posibles. Otra forma de calcular el promedio de este observable, es considerar el promedio en el tiempo del mismo, es decir, considerando que los puntos $P_{i}$ de salida de la simulación forman una trayectoria $\mathbf{\Gamma}(t)$ en el espacio de fase, puede calcularse el promedio en el tiempo, como una función de $\mathbf{\Gamma}(t)$:

\begin{equation}
\langle \mathbf{\mathcal{A}} \rangle _{t} = \frac{1}{\tau} \sum_{\tau = 1}^{\tau_{obs}} \mathbf{\mathcal{A}}(\Gamma (t))
\end{equation}

donde $\tau_{obs}$ es el número de pasos de la simulación.

Si bien estos dos promedios se definen de diferente manera, la \textbf{hipótesis egódica} nos dice que los dos son iguales. Definimos un sistema \textbf{ergódico} como un sistema para el cual, sobre un conjunto $V$ de $\Omega$, el Promedio de Tiempo\footnotemark{} de una función $\mathbf{\mathcal{A}}$ a lo largo de una trayectoria $\mathbf{\Gamma}(t)$ es igual al Promedio de Fase\footnotemark{} de $\mathbf{\mathcal{A}}$ en casi todas partes\footnotemark.

\footnotetext{$\lim_{t\to\infty} \int_{0}^{t} \mathbf{\mathcal{A}}(P,T) dt = \bar{\mathbf{\mathcal{A}}}(P)$ se llama Promedio de Tiempo de $\mathbf{\mathcal{A}}$ a lo largo de la trayectoria que pasa por $P$}
\footnotetext{El teorema de Birkhoff dice: Sea $V$ un subconjunto del Espacio de Fase $\Omega$ invariante bajo una trasformación $T:V\rightarrow{}V$ y que tiene un volumen generalizado finito. Sea  $\mathbf{\mathcal{A}}$ =  $\mathbf{\mathcal{A}}(P)$, una función de fase, definida para todos los puntos de $V$ e integrable sobre $V$. Entonces el límite $\lim_{t\to\infty} \int_{0}^{t} \mathbf{\mathcal{A}}(P,T) dt$ existe para casi todos los puntos de $V$}
\footnotetext{$\frac{1}{\mu{}(V)} \int_{V} \mathbf{\mathcal{A}}(p) dV$ se llama Promedio de Fase de $\mathbf{\mathcal{A}}$ sobre $V$ y es el promedio de la función $\mathbf{\mathcal{A}}$, calculado sobre el conjunto invariante $V$}

Esto quiere decir que podemos calcular propiedades termodinámicas a partir del Promedio de Tiempo de la propiedad de interés, sin tener que calcular el promedio del ensamble. La idea es que con tiempos suficientemente grandes, el sistema pasa por un número de estados suficientemente grande para considerar estos dos promedios iguales.

\begin{equation}
\langle \mathbf{\mathcal{A}} \rangle _{t} = \langle \mathbf{\mathcal{A}} \rangle _{ens}
\end{equation}

Existen diferentes ensambles, como por ejemplo:

\begin{description}
	\item[Ensamble Microcanónico $(N,V,E)$] Se fija el número de partículas, el volumen y la energía total.
	\item[Ensamble Canónico $(N,V,T)$] Se fija el número de partículas, el volumen y la temperatura.
	\item[Ensamble Macrocanónico $(\mu,V,T)$] Se fija el potencial químico, el volumen y la temperatura.
	\item[Ensamble Isotérmico-Isobárico $(N,P,T)$] Se fija el número de partículas, la presión y la temperatura.
	\item[Ensamble Isobárico-Isoentálpico $(N,P,H)$] Se fija el número de partículas, la presión y la entalpía.
\end{description}

%----------------------------------------------------------------------------------------
%	SECTION 3
%----------------------------------------------------------------------------------------
\section{Potenciales Interatómicos}
\label{S2_3}

El grado de semejanza a la realidad que puede tener una simulación de MD depende (entre otros) de que las fuerzas interatómicas simuladas representen las presentes en el material real. Estas fuerzas son calculadas internamente como el gradiente de una \textit{función de energía potencial}, que depende de las posiciones atómicas (Ecuaciónes \ref{eq:pot},\ref{eq:potfza}). Hay que aclarar que esta función puede responder bien a determinadas condiciones o composiciones, pero no a otras: hay que estar seguro de que el potencial es representativo para las condiciones de nuestra simulación.

\begin{eqnarray}
V(\mathbf{r}_{1},...,\mathbf{r}_{2})
\label{eq:pot}\\
\mathbf{F}_{i} = -\nabla_{\mathbf{r}_{i}}V(\mathbf{r}_{1},...,\mathbf{r}_{2})
\label{eq:potfza}
\end{eqnarray}

Existen numerosos potenciales utilizados actualmente, algunos más generales que otros y con diferente formulación. Para sistemas complejos, los potenciales se generan en forma tabular, y no se tiene una expresión analítica del mismo, debiendo usarse el que corresponda a nuestra composición y parámetros de simulación.

Se nombran dos ejemplos a continuación: el potencial \textit{Lennard-Jones} y el \textit{Modelo de Átomo Embebido (EAM)}. El último es el que se utiliza en todas las simulaciones de MD del presente trabajo.

\subsection{Potencial Lennard-Jones}
\label{SS2_3_1}

El potencial Lennard-Jones es un modelo simple que describe la interacción entre dos 	átomos neutrales, y su expresión matemática se ve en la Ecuación \ref{eq:LJ}. El valor $\varepsilon$ es la profundidad del pozo de energía potencial, $\sigma$ es la distancia a la cual el potencial es cero, $r_{m}$ es la distancia para la cual se obtiene el mínimo de energía potencial y $r$ es la distancia entre los átomos. A $r_{m}$, el potencial vale $-\varepsilon$.

\begin{equation}
V_{LJ}=
4\varepsilon \left[ \left( \frac{\sigma}{r} \right)^{12} - \left( \frac{\sigma}{r} \right)^{6} \right] = 
\varepsilon \left[ \left( \frac{r_{m}}{r} \right)^{12} -2 \left( \frac{r_{m}}{r} \right)^{6} \right]
\label{eq:LJ}
\end{equation}

El término con $r^{-12}$ es un término de repulsión asociado a la repulsión de Pauli\footnotemark{} y el término con $r^{-6}$ representa la atracción de largo alcance (fuerzas de van der Waals\footnotemark). Podemos ver gráficamente el potencial en la Figura \ref{C2:fg:LJ}.

\begin{figure}[htp]
\centering
\includegraphics[width=10cm]{Cap_2/LJ.png}
\caption[Potencial de Lennard-Jones]{Gráfica del potencial Lennard-Jones normalizado a $\varepsilon$ y $\sigma$}
\label{C2:fg:LJ}
\end{figure}

\subsection{Modelo de Átomo Embebido}
\label{SS2_3_2}

En este método, la energía potencial es función de una suma de funciones de la separación entre un átomo y sus vecinos. La energía potencial de un átomo $i$ está dada por la Ecuación \ref{C2:eq:EAM}

\begin{equation}
V_{i} = F_{\alpha}\left(\sum_{i\neq j} \rho_{\beta} (r_{ij}) \right) + \frac{1}{2} \sum_{i\neq j} \phi_{\alpha\beta}(r_{ij})
\label{C2:eq:EAM}
\end{equation}

donde $r_{ij}$ es la distancia entre los átomos $i$ y $j$, $\phi_{\alpha\beta}$ es una función potencial entre pares de átomos, $\rho_{\beta}$ es la contribución a la densidad de carga electrónica del átomo $j$ de tipo $\beta$ en la ubicación del átomo $i$ y $F_{\alpha}$ es una función que representa la energía requerida para colocar el átomo $i$ de tipo $\alpha$ en la nube de electrones.

Al definir las funciones $\phi_{\alpha\beta}(r_{ij})$, $F_{\alpha}(\rho)$ y $\rho_{\beta} (r_{ij})$ se establece el modelo para un material en particular. En el caso de nuestras simulaciones, se tiene una aleación metálica de dos componentes, lo que implica definir siete funciones: tres funciones potenciales ($\phi_{\alpha\alpha}$, $\phi_{\alpha\beta}$, $\phi_{\beta\beta}$), dos funciones integradoras ($F_{\alpha}$, $F_{\beta}$) y dos funciones de contribución a la densidad de carga electrónica ($\rho_{\alpha}$, $\rho_{\beta}$).

Las funciones se encuentran en forma tabular en un archivo de formato \textit{setfl} \citep{setfl} que es leído por lammps al indicarlo en el script de entrada. El archivo que utilizamos en nuestras simulaciones fue creado por H.W.Sheng \citep{cheng09}.

%----------------------------------------------------------------------------------------
%	SECTION 4
%----------------------------------------------------------------------------------------
\section{Termostatos y Barostatos}
\label{S2_4}


%----------------------------------------------------------------------------------------
%	SECTION 5
%----------------------------------------------------------------------------------------
%\section{Modelo mecánico}
\section{Modelo numérico}
\label{S2_5}


%-----------------------------------
%	SUBSECTION
%-----------------------------------

\subsection{EDO que resulve LAMMPS}
\label{S2_5_1} 
% Chapter Template

\chapter{MECOM 2012} % Main chapter title

\label{C3} % Change X to a consecutive number; for referencing this chapter elsewhere, use \ref{ChapterX}

\lhead{Capítulo 3. \emph{MECOM 2012}} % Change X to a consecutive number; this is for the header on each page - perhaps a shortened title

%----------------------------------------------------------------------------------------
%	SECTION 1
%----------------------------------------------------------------------------------------

\section{ASD}



%-----------------------------------
%	SUBSECTION 1
%-----------------------------------

\subsection{ASD}



%----------------------------------------------------------------------------------------
%	SECTION 2
%----------------------------------------------------------------------------------------

\section{ASD}


%----------------------------------------------------------------------------------------
%	SECTION 3
%----------------------------------------------------------------------------------------

\section{ASD}


% Chapter Template

\chapter{PANACM NANOPARTICULAS} % Main chapter title

\label{C4} % Change X to a consecutive number; for referencing this chapter elsewhere, use \ref{ChapterX}

\lhead{Capítulo 4. \emph{PANACM NANOPARTICULAS}} % Change X to a consecutive number; this is for the header on each page - perhaps a shortened title

%----------------------------------------------------------------------------------------
%	SECTION 1
%----------------------------------------------------------------------------------------

\section{ASD}


%-----------------------------------
%	SUBSECTION 1
%-----------------------------------

\subsection{ASD}


%----------------------------------------------------------------------------------------
%	SECTION 2
%----------------------------------------------------------------------------------------

\section{ASD}


%----------------------------------------------------------------------------------------
%	SECTION 3
%----------------------------------------------------------------------------------------

\section{ASD} 
% Chapter Template

\chapter{ESTUDIO DE LAS PROPIEDADES MECANICAS DE UN BMG POROSO SINTERIZADO (NANOVIDRIOS)} % Main chapter title

\label{C5} % Change X to a consecutive number; for referencing this chapter elsewhere, use \ref{ChapterX}

\lhead{Capítulo 5. \emph{ESTUDIO DE LAS PROPIEDADES MECANICAS DE UN BMG POROSO SINTERIZADO (NANOVIDRIOS)}} % Change X to a consecutive number; this is for the header on each page - perhaps a shortened title


%----------------------------------------------------------------------------------------
%	SECTION 1
%----------------------------------------------------------------------------------------

\section{Introducción}
\label{S5_1}

% Un vidrio metálico (MG), también llamado ``metal amorfo'', es una aleación metálica que posee una estructura amorfa,
% en oposición a la estructura cristalina que normalmente presentan los metales. Esto puede lograrse mediante varias técnicas,
% la mayoría de las cuales incluyen altas velocidades de temple, peque\~nos volúmenes y el control de la composición del
% material \citep{liebermann93}. Como resultado, estos materiales cuentan con algunas ventajas con respecto a los metales cristalinos:
% mejor elasticidad combinada con una alta resistencia, dureza y moldeabilidad \citep{telford04}.

% Existen dos enfoques principales para simular vidrios metálicos sometidos a deformación plástica: haciendo foco en el comportamiento
% a escala nanométrica \citep{ogata06,guan10} o utilizando mecánica del continuo \citep{malvern69}. Para el primer enfoque, las simulaciones
% con dinámica molecular (MD) son frecuentemente utilizadas \citep{allen87}. La dinámica molecular puede resolver problemas en los cuales
% interactuan muchos cuerpos (átomos), mediante la aplicación de un potencial entre pares de átomos. Así, este enfoque es útil para el
% estudio de propiedades a escala nanométrica, tal como la deformación, la tensión, la temperatura, etc.

% Las simulaciones con dinámica molecular también son útiles para identificar procesos pláticos en BMGs. La plasticidad comienza con la
% formación de zonas de transformación de tensión cortante (STZ), las cuales se nuclean, formando bandas de corte \citep{ogata06,shimizu07} a medida que la
% deformación aumenta. Las bandas de corte (SB) pueden provocar la falla frágil del material debido a la deformación heterogénea. De allí la importancia
% de prevenir o retrasar su propagación. En los metales cristalinos se procede de forma similar al trabar las dislocaciones.

Los vidrios metálicos con porosidad han sido objeto de mucho estudio en los últimos años \citep{guan13,wang10}, en un esfuerzo
por mejorar el entendimiento de su mecánica de deformación. El comportamiento en el
régimen elastoplástico puede ser controlado mediante la introducción de poros.

Las grandes deformaciones normalmente se deben, como se ha visto en capítulos anteriores, al colapso de zonas de transformación de
tensión cortante (STZ) que dan lugar a una o varias bandas de corte (SB), las cuales pueden provocar la falla frágil del material debido a la
deformación heterogénea. De allí la importancia de prevenir o retrasar su propagación.

La adición de poros en metales cristalinos reduce, como es sabido, el movimiento de dislocaciones y modifica la
deformación plástica resultante. De igual manera, los poros en vidrios metálicos limitan la propagación de bandas de corte y permiten
una deformación más homogénea. Ultimamente, se ha suscitado un gran interés al respecto, y varias opciones han sido exploradas
\citep{guan13,wang10,schuh07,liontas14}.

En el presente capítulo fabricamos muestras de BMG Cu$_{46}$ Zr$_{54}$ con porosidad
(similar a las muestras de ``nanovidrios'' en otros experimentos y simulaciones \citep{adibi13,albe13}) mediante el sinterizado de nanopartículas.
Las simulaciones con dinámica molecular nos permitirán analizar los polyedros de Voronoi, las tensiones y deformaciones atómicas,
así como también la forma en que la porosidad inicial afecta la deformación resultante de la muestra.

% En un trabajo previo hemos determinado los parámetros constitutivos del vidrio metálico Cu$_{46}$ Zr$_{54}$ como
% una función de la temperatura. Ahora presentamos resultados para un vidrio metálico de igual composición, pero
% fabricado mediante el sinterizado de nanopartículas de BMG, lo cual resulta en muestras con porosidad, similar
% a las muestras de ``nanovidrios'' en otros experimentos y simulaciones \citep{adibi13,albe13}. Para llevar a cabo las simulaciones atomísticas
% hemos utilizado el enfoque de Dinámica Molecular (MD), y el estudio incluye análisis de los polyedros de Voronoi, tensiones
% y deformaciones atómicas. Analizamos de qué forma depende la deformación en la fracción de volumen sólido (SVF), y cómo la deformación
% se distribuye a lo largo de la muestra en función de la porosidad inicial.

% Una deformación más homogénea puede ser lograda mediante el agregado de nanoinclusiones al material. Ultimamente, se ha suscitado
% un gran interés al respecto, y varias opciones han sido exploradas \citep{guan13,wang10,schuh07,liontas14}. Siguiendo trabajo previo, en el
% presente trabajo fabricamos muestras de BMG Cu$_{46}$ Zr$_{54}$ con porosidad (similar a las muestras de ``nanovidrios'' en otros
% experimentos y simulaciones \citep{adibi13,albe13}) mediante el sinterizado de nanopartículas. Las simulaciones con dinámica molecular
% nos permitirán analizar los polyedros de Voronoi, las tensiones y deformaciones atómicas, así como también la forma en que la porosidad inicial
% afecta la deformación resultante de la muestra.

%----------------------------------------------------------------------------------------
%	SECTION 2
%----------------------------------------------------------------------------------------

\section{Detalles de Simulación y Preparación de la Muestra Porosa}
\label{S5_2}

% Para este trabajo, las simulaciones con dinámica molecular fueron realizadas mediante el uso del programa LAMMPS \citep{plimpton95},
% el cual es gratuito y de código libre, tiene un muy buen manual y es computacionalmente eficiente para la simulación de sistemas
% con gran número de átomos. Por otro lado, el análisis de Voronoi y las imágenes de la muestra fueron realizadas con el programa Ovito
% \citep{stukowski10}, y otras figuras fueron graficadas con Gnuplot. Ambos programas son gratuitos y de código libre.

\begin{figure}[h!]
  \centering
  \begin{tabular} {c}
    \includegraphics[width=10cm]{Cap_5/spheres2.png}\\
    \includegraphics[width=10cm]{Cap_5/spheres3.png}\\
  \end{tabular}
  \caption[Imágenes de la muestra]{Imágenes de la muestra (a) anterior al proceso de sinterizado y (b) posterior
  al proceso de sinterizado (porosidad 13\%).}
  \label{C5:fg:sint}
\end{figure}

Para la preparación de la muestra porosa, hemos tomado como punto de partida la misma muestra utilizada con anterioridad (capítulo \ref{C3}).
% y descripta por \cite{arman10}. Se trata de una muestra prismática Cu$_{46}$ Zr$_{54}$ con un total de 160000 átomos y obtenida
% mediante una velocidad de enfriamiento de 10$^{12}$ K/s. La temperatura de transición vítrea experimental (T$_{g}$) de este vidrio metálico
% es de 696 K, y el módulo de cizalladura (o módulo de elasticidad transversal G) es 30 GPa \citep{johnson05}. 
Para describir las interacciones
entre los átomos, usamos nuevamente el potencial EAM (método de átomo embebido) \citep{daw84}. Usamos condiciones de frontera periódicas en las
tres caras, lo cual es apropiado para altas velocidades de deformación \citep{bringa05}. De esta forma, podemos simular un BMG (de mayor tamaño)
y evitar concentración de tensiones en las fronteras.

Tomamos la muestra original y la replicamos a modo de obtener una muestra aproximadamente cúbica, de 15 nm de lado. Luego, seleccionamos puntos
al azar dentro de la muestra para hacer de centro de las esferas (de 2.5 nm de radio), y removimos todo el material al exterior de dichas esferas
para simular el sinterizado de nanopartículas esféricas de material. La figura~\ref{C5:fg:sint}-(a) muestra el resultado de estas acciones. Esta
nueva muestra tiene 77888 átomos.

El procedimiento para simular el sinterizado fue de relajar la muestra a una temperatura constante de 650K, justo debajo de la temperatura
de transición vítrea, y volúmen constante durante unos pocos ps, y luego aplicar hasta 10 ps de presión compresiva (400 bar). Estos dos pasos
fueron repetidos hasta lograr las porosidades deseadas. Luego, relajamos la muestra una vez más mediante el siguiente procedimiento: temple a 
temperatura cero mediante una velocidad de enfriamiento de $6.5 \cdot 10^{14} K/s$, aplicación de un baróstato para llegar a presión
cero, calentamiento a la misma velocidad que la velocidad de enfriamiento para llegar a la temperatura de simulacion (300K) y,
finalmente, aplicación de un baróstato durante 5 ps para reducir la presión a cero mientras se mantiene la temperatura constante. La 
figura~\ref{C5:fg:sint}-(b) exhibe una de las muestras obtenidas mediante el proceso que se ha explicado.

Se prepararon muestras con distintas porosidades iniciales (3.3\%, 5.8\% y 13.1\%). Estas muestras estabilizadas fueron luego utilizadas
para realizar carga uniaxial de compresión y tracción. Todas las coordenadas atómicas fueron recalculadas cada paso, en relación con la
velocidad de deformación requerida, la cual fue de $10^9 /s$, valor apropiado para experimentos de choque compresivo.

%----------------------------------------------------------------------------------------
%	SECTION 3
%----------------------------------------------------------------------------------------

\section{Resultados}
\label{S5_3}

A continuación presentamos resultados para deformación puramente uniaxial, la cual es adecuada para realizar comparaciones con resultados de
experimentos a altas velocidades de deformación, donde las deformaciones laterales pueden ser despreciadas.

\begin{figure}[h!]
  \centering
  \begin{tabular} {c}
    \includegraphics[width=10cm]{Cap_5/SVF_strain_comp_dash.eps}\\
    \includegraphics[width=10cm]{Cap_5/SVF_strain_tens.eps}\\
  \end{tabular}
  \caption[Fracción de volumen sólido (SVF) versus deformación.]{Fracción de volumen sólido (SVF) versus deformación. (a) Compresión
  (b) Tracción. En (a), las líneas a trazos indican la deformaión a la cual los poros se cierran por completo}
  \label{C5:fg:svf}
\end{figure}

La figura~\ref{C5:fg:svf} muestra la evolución de la densidad en las muestras porosas. Para el caso de compresión, hemos agregado unas líneas a trazos
que indican el punto en el que los poros se cierran completamente, es decir, la fracción de volumen sólido es igual a 1.
Para cada porosidad, el valor de deformación $\epsilon$ correspondiente es: 3\% $\rightarrow$ 0.05, 6\% $\rightarrow$ 0.065, 13\% $\rightarrow$ 0.12.
En algunas de las figuras de esta sección aparecerán estas líneas nuevamente, a fin de observar si este evento implica cambios en el comportamiento
plástico. Para el caso de tracción, el uso de condiciones de borde periódicas evita que los poros se cierren incluso a altas
deformaciones uniaxiales, dado que no hay deformaciones laterales.

\begin{figure}[h!]
  \centering
  \begin{tabular} {c}
    \includegraphics[width=10cm]{Cap_5/Pzz_strain_comp_dash.eps}\\
    \includegraphics[width=10cm]{Cap_5/Pzz_strain_tens.eps}\\
  \end{tabular}
  \caption[Presión en el eje Z vs deformación.]{Presión en el eje Z vs deformación. (a) Compresión (b) Tracción.}
  \label{C5:fg:pzz2}
\end{figure}

\begin{figure}[h!]
  \centering
  \begin{tabular} {c}
    \includegraphics[width=10cm]{Cap_5/stress_strain_comp_dash.eps}\\
    \includegraphics[width=10cm]{Cap_5/stress_strain_tens.eps}\\
  \end{tabular}
  \caption[Tensión de von Mises vs deformación.]{Tensión de von Mises vs deformación. (a) Compresión (b) Tracción.}
  \label{C5:fg:stress}
\end{figure}

La figura~\ref{C5:fg:pzz2}-(a) grafica la presión en el eje de carga versus deformación para esfuerzos de compresión. La figura~\ref{C5:fg:stress}-(a),
a su vez, grafica el esfuerzo de von Mises versus deformación para esfuerzos de compresión.
Ambas gráficas nos indican que la presencia de porosidad promueve el inicio de la plasticidad. Los poros actúan como concentradores de tensión,
facilitando la aparición de STZs y bandas de corte. Esta plasticidad temprana comienza a
cerrar los poros, produciendo una curva donde la deformación $\epsilon$ aumenta mientras que la presión se mantiene baja.
Cuando los poros se cierran, la presión aumenta más aceleradamente, como puede apreciarse en la porción de la curva posterior
a la línea de trazos. Podemos también observar que el comportamiento de las muestras porosas luego de las líneas de trazos es
muy similar al comportamiento de la muestra sin porosidad, validando el hecho de que a ese punto ya no hay más porosidad. Basándonos en la figura,
podemos concluir que a mayor porosidad, se necesita menor esfuerzo para cerrar los poros (denotado por la altura de las líneas a trazos),
pero esto ocurre a mayores deformaciones.

En tracción las muestras se comportan diferentemente. Como ya se ha dicho los poros no se cierran en tracción, como sí lo hacían en compresión.
Mediante el análisis de algunas imágenes de la muestra, como las que aparecen en la figura~\ref{C5:fg:ss_tens},
observamos que los poros crecen a una velocidad aproximadamente constante a medida que la deformación de la muestra aumenta. 
Las figuras~\ref{C5:fg:pzz2}-(b) y \ref{C5:fg:stress}-(b) también muestran lo que parece ser flujo plástico: particularmente a 13\% porosidad,
pero similarmente a otras porosidades, luego de un cierto punto, la deformación aumenta mientras la presión se mantiene constante o
incluso disminuye.

\begin{figure}[h!]
  \centering
  \begin{tabular} {c}
    \includegraphics[width=10cm]{Cap_5/tipe3_strain_comp.eps}\\
    \includegraphics[width=10cm]{Cap_5/tipe3_strain_tens.eps}\\
  \end{tabular}
  \caption[Polyedros de Voronoi tipo 3 vs deformación.]{Polyedros de Voronoi tipo 3 vs deformación. (a) Compresión (b) Tracción.}
  \label{C5:fg:tip3}
\end{figure}

\begin{figure}[h!]
  \centering
  \begin{tabular}{c}
    \includegraphics[width=8cm]{Cap_5/13_0strain.png} \\
    \includegraphics[width=8cm]{Cap_5/13_5strain_comp.png}\includegraphics[width=8cm]{Cap_5/13_12strain_comp.png} \\
  \end{tabular}
  \caption[Coloreado de una sección de la muestra con porosidad 13\% según la deformación cortante.]{Coloreado de una sección de la muestra con
  porosidad 13\% según la deformación cortante. El coloreado fue hecho usando Ovito, el color azul siendo 0.1 o menor y el color rojo 0.3 o mayor.
  (a) Estado inicial de la muestra (b) 5\% deformación por compresión (c) 12\% deformación por compresión.}
  \label{C5:fg:ss_comp}
\end{figure}

\begin{figure}[h!]
  \centering
  \begin{tabular}{c}
    \includegraphics[width=8cm]{Cap_5/13_6strain_tens.png}\includegraphics[width=8cm]{Cap_5/13_20strain_tens.png} \\
  \end{tabular}
  \caption[Coloreado de una sección de la muestra con porosidad 13\% según la deformación cortante.]{Coloreado de una sección de la muestra con
  porosidad 13\% según la deformación cortante. El coloreado fue hecho usando Ovito, el color azul siendo 0.1 o menor y el color rojo 0.3 o
  mayor. (a) 6\% deformación por tracción (b) 20\% deformación por tracción.}
  \label{C5:fg:ss_tens}
\end{figure}

La figura~\ref{C5:fg:tip3} muestra curvas de polyedros de Voronoi versus deformación. El análisis por teselado de Voronoi es una técnica para
caracterizar el ordenamiento local en vidrios metálicos amorfos, donde cada átomo es el centro de un polyedro de Voronoi,
completado por sus vecinos más cercanos. En \cite{arman10}, los átomos de tipo 3 son identificados como indicadores de plasticidad, por eso
son de gran importancia.

La figura~\ref{C5:fg:tip3}-(a) presenta las curvas de polyedros de Voronoi para esfuerzos de compresión. El gráfico muestra una caída en el número
de los átomos tipo 3, la cual sucede luego de una fase constante. Se ha pensado en este fenómeno como un indicador del inicio de la plasticidad
\citep{arman10}. Sin embargo, nuestras curvas muestran un resultado contraintuitivo, ya que la plasticidad comienza antes en las muestras con menor
porosidad de acuerdo a nuestro análisis. Esto podría ser considerado como un indicador de que hay otros factores o procesos en juego que afectan
los resultados.

Las figuras~\ref{C5:fg:ss_comp}-(b) y (c) presentan la evolución de la deformación cortante en la muestra, para esfuerzos de compresión.
Salta inmediatamente a la vista que los poros actúan como concentradores de tensiones, pero también representan un obstáculo para la
propagación de bandas de corte \citep{wang10}. Las bandas de corte nuclean diagonalmente en el espacio entre poros, y la deformación atómica
se acumula a lo largo de estas direcciones principales por el resto de la simulación, como puede observarse en la imágen de la muestra a 
deformación 12\%. Un endurecimiento de la muestra ocurre algunos momentos previo al cierre total de los poros, tal y como a apreciado
\cite{yuan14} y puede verse en la figura~\ref{C5:fg:pzz2}-(a).

La figura~\ref{C5:fg:tip3}-(b) muestra las curvas de poledros de Voronoi para esfuerzos de tracción. En esta imágen, los átomos de tipo 3
prácticamente no varían en las muestras con porosidad, lo que implicaría que no hay formación de STZs.
Para la muestra no porosa, el número de átomos tipo 3 se vuelve aproximadamente constante luego de que se ha nucleado el poro. Esto
nos llevó a pensar que, dada las condiciones de las muestras, se facilita el movimiento de los átomos alrededor de los poros, lo cual evita
la formación de STZs fuera de los alrededores de los poros. Para apoyar esta idea, en las figuras~\ref{C5:fg:ss_tens}-(b) y (c) presentamos
la muestra coloreada con la deformación cortante. Es evidente que la deformación cortante se concentra principalmente alrededor
de los poros. Debe ser mencionado que la posición relativa entre átomos que se encuentran lejanos de los poros se mantiene aproximadamente igual.


%----------------------------------------------------------------------------------------
%	SECTION 4
%----------------------------------------------------------------------------------------

\section{Conclusiones}
\label{S5_4}

Se realizaron simulaciones de Dinámica Molecular (MD) en una muestra porosa del vidrio metálico Cu$_{46}$ Zr$_{54}$, aplicando esfuerzos
de compresión y tracción. Los resultados bajo deformación fueron comparables a aquellos encontrados en la literatura \citep{yuan14} para
la compresión de muestras porosas de monocristales de cobre. Esto puede ser considerado como una validación del proceso de sinterizado 
utilizado para la preparación de las muestras.

Con carga compresiva, los poros facilitan la plasticidad actuando como concentradores de tensiones, pero también retrasan
la formación de zonas de transformación de tensión cortante (STZs) y su posible unión en una banda de corte (SB), para el material lejano a los poros.
Los resultados también exhiben un endurecimiento de la muestra al cerrarse los poros, similarmente a lo que ocurre en el caso no poroso.

Con carga de tracción y deformación puramente uniaxial, los poros no cierran y concentran flujo plástico alrededor de ellos, a su vez
que también impiden la formación de STZs y bandas de corte.

% Estudios futuros incluirán un análisis de Voronoi profundizado y la simulación de muestras más grandes con topologías de porosidad diferentes.

 
% Chapter Template

\chapter{CONCLUSIONES} % Main chapter title

\label{C6} % Change X to a consecutive number; for referencing this chapter elsewhere, use \ref{ChapterX}

\lhead{Capítulo 6. \emph{CONCLUSIONES}} % Change X to a consecutive number; this is for the header on each page - perhaps a shortened title

%----------------------------------------------------------------------------------------
%	SECTION 1
%----------------------------------------------------------------------------------------

\section{Generales}

Se realizaron simulaciones atomísticas del comportamiento mecánico de vidrios metálicos volumétricos (BMGs) bajo diferentes modos de carga y con diferentes modificaciones a su composición haciendo uso de simulaciones de dinámica molecular (MD). En el \cref{C3} se estudia una matriz amorfa de cobre-circonio ($Cu_{46}Zr_{54}$) bajo tensiones de tracción y compresión y se observan los efectos de la temperatura en los parámetros constitutivos asíi como en la respuesta mecánica. Luego, en el \cref{C4}, se estudia la estabilidad de nanopartículas cristalinas en la matriz amorfa previamente caracterizada y se comprara el comportamiento mecánico con el aquél del \cref{C3}. Por último, en el \cref{C5}, nos enfocamos en analizar el impacto sobre el comportamiento mecánico de una matriz porosa al graduar la porosidad a diferentes niveles.

Como podemos observar en las curvas de tensión-deformación del \cref{C3}, el comportamiento tanto en tracción como en compresión de la muestra original presenta una variación suave en función de la temperatura, lo que vemos reflejado en los buenos ajustes obtenidos con un decremento exponencial con la temperatura (Figuras \ref{C3:fg:youngVsT}, \ref{C3:fg:peakVMisesVsT}, \ref{C3:fg:peakVMises1218VsT} y \ref{C3:fg:fitDosTercios}), típico de procesos activados termicamente. Se observa un claro apartamiento del caso a 900 K del resto de las simulaciones, donde el incremento de la temperatura produce un decremento considerable del módulo elástico y de la tensión de von Mises. Esto es de esperar al ser una temperatura muy superior a la temperatura de transición vítrea (696 K). Cabe destacar que en todos los casos se encuentra una asimetría en tracción y compresión, y los valores se encuentran más próximos sólo en el caso del módulo de Young (\fref{C3:fg:youngVsT}). Las tensiones máximas son mucho mayores en compresión.

Si bien un indicio de banda de corte (SB) puede observarse en la \fref{C3:fg:SBs}, no se observaron SBs claramente definidas como las que vemos en la \fref{C1:fg:shearbands}, lo que es de esperarse dado que nuestra muestra fue generada con una velocidad de enfriamiento muy elevada. Al no contar con la presencia de SBs, la identificación del comienzo de la plasticidad es muy compleja. Si bien podemos observar zonas de transformación del esfuerzo de corte (STZs), para un correcto análisis de las mismas sería necesario realizar simulaciones de una duración mucho mayor a las actuales.

Al estudiar los efectos de la temperatura sobre nanopartículas embebidas, los ajustes de difusividad en función de la temperatura pudieron realizarse para el caso de partícula Cu-FCC con $T \geq 500 K$ donde existen desplazamientos atómicos lo suficintemente importantes (\fref{C4:fg:msd500_800_FCC}). Esto puede interpretarse como el hecho de que la partícula es estable a temperaturas menores a 500 K, es decir, no se disuelve en la matriz. Sim embargo, a temperaturas más elevadas los átomos difunden y se pierde la estructura cristalina y con esto la interfase definida entre material cristalino y amorfo. Observando las curvas de desplazamientos para el caso de la partícula CuZr-B2 (Figuras \ref{C4:fg:msd10_400_B2} y \ref{C4:fg:msd500_800_B2}) vemos una pendiente cercana a cero en todos los casos, lo que dificulta el ajuste de difusividad. Si bien esto puede interpretarse como la estabilidad de la partícula, hay que tener en cuenta que los desplazamientos iniciales en iguales condiciones son mucho mayores para la partícula CuZr-B2 que para Cu-FCC (Figuras \ref{C4:fg:msd500_800_FCC} y \ref{C4:fg:msd500_800_B2}). Es en el proceso de calentamiento de la muestra desde 300 K a la temperatura de experimento (primeros 4 ps de simulación) que se producen estos desplazamientos iniciales (Figuras \ref{C4:fg:heating500_800_FCC} y \ref{C4:fg:heating500_800_B2}).

A al hora de realizar cargas sobre la matriz con inclusiones cristalinas se aplicaron condiciones de frontera periódicas en tres dimensiones, y la ausencia de superficies libres deja sólo a la nanopartícula como probable concentrador de esfuerzos para promover la nucleación de STZs y así desencadenar las bandas de corte. Cabe destacar que esta concentración de esfuerzo no es suficiente para desencadenar SBs en ningún caso (como sí podemos observar en otros trabajos \citep{albe13,brink15,adibi13,adibi14}) y si bien en todos los casos de tracción se nuclean poros (lo cual no es sorprendente luego de haberse encontrado con el mismo fenómeno al estudiar la matriz amorfa), los mismos nuclean en zonas alejadas de la nanopartícula.

Si bien las curvas de esfuerzo-deformación (Figuras \ref{C4:fg:fcc_vm_tension} a \ref{C4:fg:b2_vm_compression}) son claramente similares al caso sin nanopartícula, encontramos que a excepción de un caso (partícula CuZr-B2 a 200 K) se produce un retardo en la nucleación de un poro al traccionar la muestra. En este caso nombrado, la nucleación se produce antes que en la muestra original. El análisis de Voronoi no muestra diferencias significativas entre las muestras con y sin inclusión de nanopartícula. 

Los resultados de deformar la matriz original con el agregado de porosidad fueron comparables a aquellos encontrados en la literatura \citep{yuan14} para la compresión de muestras porosas de monocristales de cobre. Esto puede ser considerado como una validación del proceso de sinterizado utilizado para la preparación de las muestras.

Con carga compresiva, los poros facilitan la plasticidad actuando como concentradores de tensiones, pero también retrasan la formación de zonas de transformación de tensión cortante (STZs) y su posible unión en una banda de corte (SB), para el material lejano a los poros. Los resultados también exhiben un endurecimiento de la muestra al cerrarse los poros, similarmente a lo que ocurre en el caso no poroso. Con carga de tracción y deformación puramente uniaxial, los poros no cierran y concentran flujo plástico alrededor de ellos, a su vez que también impiden la formación de STZs y bandas de corte.


%----------------------------------------------------------------------------------------
%	SECTION 2
%----------------------------------------------------------------------------------------

\section{Trabajos Futuros}


- Elementos  Finitos.
- Voronoi más particular.
- Generar muestras a menor quenching rate.
- Muestras porosas de mayor tamaño para lograr topologías diferentes.
- Muestras de mayor tamaño con mayor número de nanopartículas.

%% CAP 3

%Atomistic simulations of bulk metallic glasses (BMGs) mechanical behavior under tension and compression were performed using molecular dynamics (MD) simulations. 

%The increase of sample temperature produces a considerable decrease of the samples elastic modulus. The same applies to maximum von Mises stress. It is observed that the elastic modules are practically the same under tension or compression at different temperatures, but the maximum stress in compression is much higher. The behavior with temperature can be adjusted reasonably well with an exponential decay with temperature, typical of thermal activated phenomena.

%No shear bands are observed, which is to be expected given that our glass was generated with very high quenching rates. Since no shear bands are observed in our simulations, the identification of plasticity is complex. Surely there are shear areas, "shear transformation zones" (STZ), composed of a few atoms that experience high shear stresses. The identification of these areas requires a very detailed observation of the sample, involving much longer simulations than those used here. An alternative to study plasticity is the examination of Voronoi polyhedra, which can help to identify these areas. Such studies are in progress.

%In the future, using more powerful computational resources than available for this work, we plan to create samples with quenching rates orders of magnitude slower, with the aim to observe the possible formation of shear bands.

%A detailed understanding of the influence of temperature, quenching rates, etc., in the mechanical properties of metallic glasses will allow obtaining necessary properties for their application in new technologies, including applications under extreme conditions, such as aerospace missions or materials in nuclear reactors. Studies like the one presented here will contribute to this understanding and accelerate novel material development. 

%% CAP 4

%Estudiamos un BMG con una nanopartícula cristalina como inclusión. Consideramos un vidrio CuZr, y una nanopartícula de Cu pura con un radio de 2 nm. Ésto implica una fracción en volumen que varía desde 1.15\% a 10 K hasta 1.12\% a 800 K como resultado del aumento del volumen inicial de la muestra con la temperatura. Una situación similar fue explorada recientemente por Albe et al. \citep{albe13}. Aquí, nos centramos en los efectos de la temperatura, e inicialmente estudiamos la estabilidad debajo de los 400 K, indicando que la nanopartícula es bastante estable a esas temperaturas. A temperaturas mayores, la difusividad en sólo algunos ns trae consigo la pérdida de una interfaz nítida entre la nanopartícula y la matriz.

% En nuestras simulaciones, se aplicaron condiciones de frontera periódicas en tres dimensiones, y la ausencia de superficies libres deja sólo a la nanopartícula como probable concentrador de esfuerzo para promover la nucleación de STZs, y así desencadenar las bandas de corte, en la interfaz entre la matriz y la nanopartícula. Sin embargo, éste no fue el caso. Las curvas de esfuerzo-deformación son claramente similares al caso sin nanopartícula, a excepción de un retardo en la nucleación de un poro bajo tracción para la muestra con una nanopartícula.

% El análisis de Voronoi no muestra diferencias significativas entre las muestras con y sin inclusión de nanopartícula. Un estudio futuro y más detallado es requerido para diferentes modos de carga y temperaturas. Estudios futuros también podrían repetir estos experimentos con inclusiones de CuZr con una estructura cristalina B2, como podemos encontrar en algunas experiencias \citep{wei14,kuo14}.

%% CAP 5

%Se realizaron simulaciones de Dinámica Molecular (MD) en una muestra porosa del vidrio metálico Cu$_{46}$ Zr$_{54}$, aplicando esfuerzos de compresión y tracción. Los resultados bajo deformación fueron comparables a aquellos encontrados en la literatura \citep{yuan14} para la compresión de muestras porosas de monocristales de cobre. Esto puede ser considerado como una validación del proceso de sinterizado utilizado para la preparación de las muestras.

%Con carga compresiva, los poros facilitan la plasticidad actuando como concentradores de tensiones, pero también retrasan la formación de zonas de transformación de tensión cortante (STZs) y su posible unión en una banda de corte (SB), para el material lejano a los poros. Los resultados también exhiben un endurecimiento de la muestra al cerrarse los poros, similarmente a lo que ocurre en el caso no poroso.

%Con carga de tracción y deformación puramente uniaxial, los poros no cierran y concentran flujo plástico alrededor de ellos, a su vez que también impiden la formación de STZs y bandas de corte.

% Estudios futuros incluirán un análisis de Voronoi profundizado y la simulación de muestras más grandes con topologías de porosidad diferentes. 
%% Chapter Template

\chapter{CONCLUSIONES} % Main chapter title

\label{C7} % Change X to a consecutive number; for referencing this chapter elsewhere, use \ref{ChapterX}

\lhead{Capítulo 7. \emph{CONCLUSIONES}} % Change X to a consecutive number; this is for the header on each page - perhaps a shortened title

%----------------------------------------------------------------------------------------
%	SECTION 1
%----------------------------------------------------------------------------------------

\section{Generales}


\subsection{ASD}

%----------------------------------------------------------------------------------------
%	SECTION 2
%----------------------------------------------------------------------------------------

\section{Trabajos Futuros}
 

%----------------------------------------------------------------------------------------
%	THESIS CONTENT - APPENDICES
%----------------------------------------------------------------------------------------

\addtocontents{toc}{\vspace{2em}} % Add a gap in the Contents, for aesthetics

\appendix % Cue to tell LaTeX that the following 'chapters' are Appendices

% Include the appendices of the thesis as separate files from the Appendices folder
% Uncomment the lines as you write the Appendices

% Appendix A

\chapter{LAMMPS} % Main appendix title

\label{AA} % For referencing this appendix elsewhere, use \ref{AppendixA}

\lhead{Anexo A. \emph{LAMMPS}} % This is for the header on each page - perhaps a shortened title

Pequeña introducción acá

\section{Instalación}
\label{AA_1}

Instrucciones de instalación acá

\section{Automatización}
\label{AA_2}

Instrucciones de scripts acá

\section{Datos de salida}
\label{AA_3}

Instrucciones para análisis de datos de salida -> herramientas estadísticas.

% Appendix Template

\chapter{OVITO} % Main appendix title

\label{AB} % Change X to a consecutive letter; for referencing this appendix elsewhere, use \ref{AppendixX}

\lhead{Anexo B. \emph{OVITO}} % Change X to a consecutive letter; this is for the header on each page - perhaps a shortened title

Ciertas imágenes (por ejemplo las figuras \ref{C4:fg:fcc_tension_bmg_10K} y \ref{C5:fg:ss_comp}) fueron realizadas utilizando el software Ovito \citep{stukowski10}. En el presente apéndice explicaremos cómo obtener dichas imágenes y daremos recomendaciones para mejorar el uso del software. Como condición previa es necesario haber instalado y ejecutado LAMMPS (Apéndice \ref{AA}), ya que Ovito trabaja, entre otras cosas, con los archivos de salida de éste.

\section{Instalación}
\label{AB_1}

Ovito es una software libre para la visualización. Es posible descargarlo para diversas plataformas en el \href{http://www.ovito.org/index.php/download}{sitio web}. Siendo usuarios linux, explicaremos únicamente para un sistema operativo tipo linux.

La instalación de Ovito es considerablemente más sencilla que la de LAMMPS. Basta con descomprimir el archivo descargado y ejecutar \textit{bin/ovito}. Sin embargo puede ser útil copiar las carpetas \textit{bin}, \textit{lib} y \textit{share} a \textit{usr/local/} para poder acceder a los ejecutables a través de línea de comando sin tener que acceder a la carpeta descargada.

En ocasiones puede ser útil descargar el código fuente, puesto que podemos contar con actualizaciones del código de Ovito para las cuales tendríamos que esperar hasta que se presente la siguiente versión estable. Para ello, accedemos al \href{http://sourceforge.net/p/ovito/git/ci/master/tree/}{repositorio Git en Sourceforge.net}, descargamos el archivo y seguimos las \href{http://www.ovito.org/manual/development.build_linux.html}{instrucciones}.

\begin{lstlisting}
sudo apt-get install build-essential git cmake-curses-gui qt5-default libcgal-dev libboost-dev \
		     libqt5scintilla2-dev libavcodec-dev libavdevice-dev libavfilter-dev \
		     libavformat-dev libavresample-dev libavutil-dev libswscale-dev libnetcdf-dev \
		     libhdf5-dev libhdf5-serial-dev libbotan1.10-dev libmuparser-dev python3-dev \
		     libboost-python-dev python3-sphinx python3-numpy xsltproc docbook-xml \
		     docbook-xsl docbook-xsl-doc-html doxygen
                     
cd ovito
mkdir build
cd build
cmake -DOVITO_BUILD_DOCUMENTATION=ON \
      -DCMAKE_BUILD_TYPE=Release \
      -DPYTHON_INCLUDE_DIR=/usr/include/python3.4m \
      -DPYTHON_LIBRARY=/usr/lib/x86_64-linux-gnu/libpython3.4m.so \
      -DBoost_PYTHON_LIBRARY_RELEASE=/usr/lib/x86_64-linux-gnu/libboost_python-py34.so \
      -DBoost_PYTHON_LIBRARY_DEBUG=/usr/lib/x86_64-linux-gnu/libboost_python-py34.so \
      ..
      
make -j4
\end{lstlisting}


\section{Análisis de archivos de LAMMPS}
\label{AB_2}

Abrimos ovito desde la terminal. A continuación nos aparece una ventana como la de la figura \ref{AB:fg:ovitoScreen}

\begin{figure}[h!]
  \centering
  \includegraphics[width=10cm]{Cap_5/spheres2.png}
  \caption[Imágenes de la muestra]{Imágenes de la muestra.}
  \label{C5:fg:sint}
\end{figure}

\section{Automatización}
\label{AB_3}

Instrucciones de scripts acá


% Appendix Template

\chapter{SCRIPTS DE SHELL} % Main appendix title

\label{AC} % Change X to a consecutive letter; for referencing this appendix elsewhere, use \ref{AppendixX}

\lhead{Anexo C. \emph{SCRIPTS DE SHELL}} % Change X to a consecutive letter; this is for the header on each page - perhaps a shortened title

ASD
% Appendix Template

\chapter{SCRIPTS} % Main appendix title

\definecolor{mygreen}{rgb}{0,0.6,0}
\definecolor{mygray}{rgb}{0.5,0.5,0.5}
\definecolor{mymauve}{rgb}{0.58,0,0.82}

\lstset{ %
  backgroundcolor=\color{white},   % choose the background color; you must add \usepackage{color} or \usepackage{xcolor}
  %basicstyle=\footnotesize,        % the size of the fonts that are used for the code
  breakatwhitespace=false,         % sets if automatic breaks should only happen at whitespace
  breaklines=true,                 % sets automatic line breaking
  captionpos=b,                    % sets the caption-position to bottom
  commentstyle=\color{mygreen},      % comment style
  %deletekeywords={...},            % if you want to delete keywords from the given language
  %escapeinside={\%*}{*)},          % if you want to add LaTeX within your code
  extendedchars=true,              % lets you use non-ASCII characters; for 8-bits encodings only, does not work with UTF-8
  frame=single,	                   % adds a frame around the code
  keepspaces=true,                 % keeps spaces in text, useful for keeping indentation of code (possibly needs columns=flexible)
  keywordstyle=\color{blue},       % keyword style
  %otherkeywords={*,...},           % if you want to add more keywords to the set
  numbers=left,                    % where to put the line-numbers; possible values are (none, left, right)
  numbersep=5pt,                   % how far the line-numbers are from the code
  numberstyle=\tiny\color{mygray},   % the style that is used for the line-numbers
  rulecolor=\color{black},         % if not set, the frame-color may be changed on line-breaks within not-black text (e.g. comments (green here))
  showspaces=false,                % show spaces everywhere adding particular underscores; it overrides 'showstringspaces'
  showstringspaces=false,          % underline spaces within strings only
  showtabs=false,                  % show tabs within strings adding particular underscores
  stepnumber=2,                    % the step between two line-numbers. If it's 1, each line will be numbered
  stringstyle=\color{mymauve},       % string literal style
  tabsize=4,	                   % sets default tabsize to 2 spaces
  %title=\lstname                   % show the filename of files included with \lstinputlisting; also try caption instead of title
}


\label{AD} % Change X to a consecutive letter; for referencing this appendix elsewhere, use \ref{AppendixX}

\lhead{Anexo D. \emph{SCRIPTS}} % Change X to a consecutive letter; this is for the header on each page - perhaps a shortened title

\section{Scripts de secuenciación de tareas}

Suele ser útil, por diversas cuestiones, realizar scripts que ejecuten tareas automáticamente. Ya sea debido a la falta de tiempo o la falta de conexión remota, es más eficiente comenzar varias simulaciones de una sóla vez que comenzar varias veces una simulación. Por ello nos hemos servido de scripts que realizan la tarea de comenzar simulaciones por nosotros. Un ejemplo de estos scripts es el siguiente:

\lstinputlisting[language=sh]{Scripts/script_secuenciacion_1}

En este script se indica que se deben ejecutar tres simulaciones en ubicaciones distintas de la computadora, una después de la otra. ¿Qué sucede si, por ejemplo, el script anterior está ejecutándose y algún usuario desea iniciar una nueva simulación? Pueden ocurrir dos cosas:

\begin{itemize}
 \item que el usuario no esté enterado de que los recursos de la computadora estén siendo utilizados y que inicie una nueva simulación, lo cual hará que las dos instancias de lammps peleen por los recursos, provocando una pérdida de velocidad y, en casos extremos, la interrupción de una de las simulaciones.
 \item que el usuario sepa de la simulación en curso (ya sea porque le hayan avisado o porque haya revisado manualmente el uso de recursos con un comando como \textit{``ps -e''}) y no inicie su nueva simulación.
\end{itemize}

En este último caso, el nuevo usuario ha postergado su simulación hasta que la ejecución del script en curso finalice y sólo entonces podrá comenzar su simulación. Pero el nuevo usuario no sabrá con certeza cuándo terminará la ejecución del script para poder comenzar su simulación lo antes posible. Esto puede ser especialmente crítico cuando no se cuente con conexión remota (como para monitorear continuamente en espera de la finalización del script) y el tiempo es un factor relevante (por ejemplo, cuando se recibe permiso para el uso de computadoras de gran poder de procesamiento en institutos de investigación).

Es por esto que se comenzó el desarrollo de un programa que se ejecutaría en segundo plano y que implementara una \textit{cola} de comandos de simulación que pudiese actualizarse dinámicamente. De esta forma, no se pierde tiempo entre simulaciones y cada usuario que quiera realizar una simulación sencillamente debe agregar el comando a la cola.

Esto no es difícil de realizar. La dificultad reside en contemplar los casos por los que un comando pueda no terminar de ejecutarse. Así como no siempre se cuenta con el tiempo ni la conexión como para monitorear la computadora en espera de la finalización de un script, también es poco práctico tener que realizar este monitoreo para verificar que el comando ejecutado haya finalizado correctamente, o relanzar dicho comando en caso de que haya sido interrumpido (debido a, por ejemplo, un corte de luz). Aunque se han hecho avances al respecto, es en esta parte en la que el desarrollo a quedado incompleto.

El siguiente código corresponde al inicio del programa y a la ejecución de los comandos en la cola (lenguaje de programación C++):

\lstinputlisting[language=C++]{Scripts/simuStack/simuStack.cpp}

El código hace uso de tres scripts. Uno de ellos verifica que el estado del programa y otro verifica el estado del último comando.

\lstinputlisting[language=bash]{Scripts/simuStack/Scripts/checkPid}
\lstinputlisting[language=bash]{Scripts/simuStack/Scripts/updateRestart}

\section{Scripts de simulación}

Para realizar las simulaciones de los capítulos \ref{C4} y \ref{C5} se realizó bastante trabajo en los scripts de LAMMPS, resultando en código más complejo que el introducido en el Apéndice \ref{AA}.

\lstset{ %
  backgroundcolor=\color{white},   % choose the background color; you must add \usepackage{color} or \usepackage{xcolor}
  %basicstyle=\footnotesize,        % the size of the fonts that are used for the code
  breakatwhitespace=false,         % sets if automatic breaks should only happen at whitespace
  breaklines=true,                 % sets automatic line breaking
  captionpos=b,                    % sets the caption-position to bottom
  commentstyle=\color{mygreen},      % comment style
  language=bash,
  deletekeywords={type,echo},            % if you want to delete keywords from the given language
  %escapeinside={\%*}{*)},          % if you want to add LaTeX within your code
  extendedchars=true,              % lets you use non-ASCII characters; for 8-bits encodings only, does not work with UTF-8
  frame=single,	                   % adds a frame around the code
  keepspaces=true,                 % keeps spaces in text, useful for keeping indentation of code (possibly needs columns=flexible)
  keywordstyle=\color{blue},       % keyword style
  %otherkeywords={*,...},           % if you want to add more keywords to the set
  numbers=left,                    % where to put the line-numbers; possible values are (none, left, right)
  numbersep=5pt,                   % how far the line-numbers are from the code
  numberstyle=\tiny\color{mygray},   % the style that is used for the line-numbers
  rulecolor=\color{black},         % if not set, the frame-color may be changed on line-breaks within not-black text (e.g. comments (green here))
  showspaces=false,                % show spaces everywhere adding particular underscores; it overrides 'showstringspaces'
  showstringspaces=false,          % underline spaces within strings only
  showtabs=false,                  % show tabs within strings adding particular underscores
  stepnumber=2,                    % the step between two line-numbers. If it's 1, each line will be numbered
  stringstyle=\color{mymauve},       % string literal style
  tabsize=4,	                   % sets default tabsize to 2 spaces
  %title=\lstname                   % show the filename of files included with \lstinputlisting; also try caption instead of title
}

El siguiente código es uno de los usados en el capítulo \ref{C4}. En este se genera un cristal de CuZr tipo B2.

\lstinputlisting{Scripts/script_simulacion_1}

El siguiente código es uno de los usados en el capítulo \ref{C5}. En este se crean partículas esféricas de material y se sinterizan.

\lstinputlisting{Scripts/script_simulacion_2}

\section{Scripts de análisis de datos}

\lstset{ %
  backgroundcolor=\color{white},   % choose the background color; you must add \usepackage{color} or \usepackage{xcolor}
  %basicstyle=\footnotesize,        % the size of the fonts that are used for the code
  breakatwhitespace=false,         % sets if automatic breaks should only happen at whitespace
  breaklines=true,                 % sets automatic line breaking
  captionpos=b,                    % sets the caption-position to bottom
  commentstyle=\color{mygreen},      % comment style
  %deletekeywords={...},            % if you want to delete keywords from the given language
  %escapeinside={\%*}{*)},          % if you want to add LaTeX within your code
  extendedchars=true,              % lets you use non-ASCII characters; for 8-bits encodings only, does not work with UTF-8
  frame=single,	                   % adds a frame around the code
  keepspaces=true,                 % keeps spaces in text, useful for keeping indentation of code (possibly needs columns=flexible)
  keywordstyle=\color{blue},       % keyword style
  %otherkeywords={*,...},           % if you want to add more keywords to the set
  numbers=left,                    % where to put the line-numbers; possible values are (none, left, right)
  numbersep=5pt,                   % how far the line-numbers are from the code
  numberstyle=\tiny\color{mygray},   % the style that is used for the line-numbers
  rulecolor=\color{black},         % if not set, the frame-color may be changed on line-breaks within not-black text (e.g. comments (green here))
  showspaces=false,                % show spaces everywhere adding particular underscores; it overrides 'showstringspaces'
  showstringspaces=false,          % underline spaces within strings only
  showtabs=false,                  % show tabs within strings adding particular underscores
  stepnumber=2,                    % the step between two line-numbers. If it's 1, each line will be numbered
  stringstyle=\color{mymauve},       % string literal style
  tabsize=4,	                   % sets default tabsize to 2 spaces
  %title=\lstname                   % show the filename of files included with \lstinputlisting; also try caption instead of title
}

Los scripts de análisis de datos son los que más trabajo han recibido y, coincidentemente, los que más trabajo pueden ahorrar. Obtener gráficas \textit{``publication ready''}, realizar fits de datos, calcular polyedros de Voronoi son ejemplos de lo que se puede hacer a través de scripts.

El siguiente código para Gnuplot realiza un fit de los datos en el archivo \textit{``18StressValues\_fit''}, calcula el valor R cuadrado del fit y obtiene un gráfico \textit{``publication ready''}:

\lstinputlisting[language=Gnuplot]{Scripts/script_analisis_1}

El siguiente código en Python para Ovito obtiene la deformación atómica promedio de cada tipo de polyedro de Voronoi:

\lstinputlisting[language=Python]{Scripts/script_analisis_2.py}

\addtocontents{toc}{\vspace{2em}} % Add a gap in the Contents, for aesthetics

\backmatter

%----------------------------------------------------------------------------------------
%	BIBLIOGRAPHY
%----------------------------------------------------------------------------------------

\label{Bibliography}

\lhead{\emph{Bibliography}} % Change the page header to say "Bibliography"

\bibliographystyle{unsrtnat} % Use the "unsrtnat" BibTeX style for formatting the Bibliography

\bibliography{Bibliography} % The references (bibliography) information are stored in the file named "Bibliography.bib"

\end{document}  

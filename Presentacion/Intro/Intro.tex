%%%%%%%%%%%%%%%%%%%%%%%%%%%%%%%%%%%%%%%%%%%%%%%%%%%%%
%		INTRODUCCION
%%%%%%%%%%%%%%%%%%%%%%%%%%%%%%%%%%%%%%%%%%%%%%%%%%%%%

\section[Introducci\'on]{Introducci\'on}
\subsection{Vidrios Met\'alicos}

\tikzstyle{every picture}+=[remember picture]

\begin{frame}
\frametitle{Vidrios Met\'alicos (Metal Amorfo)}

\tikzstyle{na} = [baseline=-.5ex]

\centering
\tikz[baseline]{\node[fill=gray!20,anchor=base] (t1){Vidrio};} \tikz[baseline]{\node[fill=gray!20,anchor=base] (t2){Met\'alico};}

\vspace{0.2cm}

\begin{columns}
  \begin{column}{0.48\paperwidth}
    \tikz[na]{\node[coordinate] (n1){};}
    \centering
    \begin{block}{Vidrio}
      \centering Estructura amorfa (\scriptsize{Sílice - SiO$_{2}$}\normalsize{)}
      
      \vspace{0.2cm}
      \includegraphics[width=3cm]{Cap_1/500px-Silica.png}
    \end{block}
  \end{column}
  \begin{column}{0.48\paperwidth}
    \tikz[na]{\node[coordinate] (n2) {};}
    \centering
    \begin{block}{Met\'alico}
      \centering Aleaci\'on met\'alica:\\
      Cu, Ni, Fe, Au, Zr, Be, La, Pd, Ti\\
      
      \vspace{0.2cm}
      Aunque tambi\'en puede contener no metales en baja concentraci\'on
    \end{block}
  \end{column}
\end{columns}

\begin{tikzpicture}[overlay]
        \path[->] (t1) edge [bend right] (n1);
        \path[->] (t2) edge [bend left] (n2);
\end{tikzpicture}

% \item Sus propiedades sobresalientes los hacen candidatos para aplicaciones modernas y de alta tecnolog\'ia. Material avanzado
\end{frame}

\definecolor{goodTitle}{HTML}{4D9A63}
\definecolor{goodDesc}{HTML}{A0DB8E}
\definecolor{alertTitle}{HTML}{FF2222}
\definecolor{alertDesc}{HTML}{FF9090}

\newcommand{\defaultBlocks}{
  \setbeamercolor{block title}{fg=white, bg=hsrmWarmGreyDark}
  \setbeamercolor{block body}{parent=palette secondary}
  \setbeamercolor{block title example}{fg=white, bg=hsrmSec1Dark}
  \setbeamercolor{block body example}{fg=white, bg=hsrmSec1}
  \setbeamercolor{block title alerted}{fg=white, bg=hsrmRedDark}
  \setbeamercolor{block body alerted}{fg=white, bg=hsrmRed}
}

\newcommand{\ventaja}[2]{
  \setbeamercolor{block title}{bg=goodTitle,fg=white}%
  \setbeamercolor{block body}{bg=goodDesc,fg=black}%
  \only<#1>{
    \begin{block}{Ventaja}%
    #2
    \end{block}%
   }
  \defaultBlocks%
}

\newcommand{\desventaja}[2]{
  \setbeamercolor{block title}{bg=alertTitle,fg=white}%
  \setbeamercolor{block body}{bg=alertDesc,fg=black}%
  \only<#1>{
    \begin{block}<#1>{Desventaja}%
    #2
    \end{block}%
  }
  \defaultBlocks%
}


\begin{frame}
\frametitle{Propiedades}
\begin{block}{¿Por qu\'e atrae el inter\'es de investigadores?}
 Combina propiedades de cer\'amicas y de metales a escala nanom\'etrica, resultando en un material de propiedades \'unicas
\end{block}
 
\ventaja{1}{{\begin{itemize}
             \item Alta dureza
             \item Resistencia al desgaste y la abrasi\'on
             \item Gran resistencia mec\'anica
             \item Alta resiliencia
            \end{itemize}
            }}

\end{frame}

\begin{frame}
\frametitle{Propiedades} 

\ventaja{1}{{\begin{itemize}
             \item Ausencia de efectos adversos debidos a fronteras de granos (resistencia a la corrosi\'on)
            \end{itemize}
            }}
\desventaja{1}{{\begin{itemize}
		\item Alto costo y grandes limitaciones de fabricaci\'on
		\item Gran p\'erdida de ductilidad ante la aparici\'on de bandas de corte
	      \end{itemize}
	      }}
            
\end{frame}

\begin{frame}
 \frametitle{Fabricación}
 \vspace{-0.15cm}
 \begin{block}{}
    \textit{Te\'oricamente, todo l\'iquido podr\'ia convertirse en vidrio a velocidades de enfriamiento suficientemente altas y temperaturas suficientemente bajas evitando el proceso de cristalizaci\'on} (Turnbull et al, 1961)
 \end{block}
 
 %Las t\'ecnicas de fabricaci\'on act\'uan sobre la composici\'on, el volumen y la velocidad de enfriamiento
 \begin{textblock*}{10cm}(1.4cm,4.9cm)
  \begin{columns}
    \begin{column}{4cm}
      \begin{figure}
	\includegraphics[width=4cm]{Cap_1/melt_spinning_B.png}
      \end{figure} 
      \begin{textblock*}{4cm}(1.3cm,8cm)
	\scriptsize{Proceso de \textit{Melt Spinnning}}
      \end{textblock*}
    \end{column}
    \begin{column}{6cm}
    \begin{alertblock}{Espesor}
	Los vidrios met\'alicos volum\'etricos, o \textit{Bulk Metallic Glasses} en ingl\'es, tienen una secci\'on transversal de por lo menos algunos mil\'imetros.
    \end{alertblock}
    \end{column}
  \end{columns}
 \end{textblock*}
 \begin{textblock*}{12.6cm}(0.5cm,9.2cm)
  \scriptsize{Turnbull, D. and Cohen, M., \textit{J. Chem. Phys.}, \textbf{34(1)}, 120-125 (1961)}
  \end{textblock*}
\end{frame}

\begin{frame}
 \frametitle{Mec\'anica de deformaci\'on}
 \begin{textblock*}{11.8cm}(0.5cm,1.8cm)
  \begin{block}{Bandas de corte}
  Concentraci\'on de deformaci\'on en bandas estrechas, llamadas \textbf{bandas de corte}.
  El crecimiento de estas bandas puede causar la fractura fr\'agil del material (Schuh et al, 2007).
  \end{block}
 \end{textblock*}
 \vspace{-0.7cm}
 \begin{textblock*}{\textwidth}(1cm,4.3cm)
  \begin{figure}
  \centering
  \includegraphics[width=5.5cm]{Cap_1/shearbands.png}
  \end{figure}
 \end{textblock*}
\begin{textblock*}{\textwidth}(1cm,8.2cm)
 \centering
 \scriptsize{Muestra de BMG bajo tracci\'on uniaxial. Adaptado de Albe et al, 2013}
\end{textblock*}

 \begin{textblock*}{12.6cm}(0.5cm,8.9cm)
 \scriptsize{Schuh, C., Hufnagel, T., and Ramamurty, U., \textit{Acta. Mater.}, \textbf{55(12)}, 4067-4109 (2007)} \\
 \scriptsize{Albe, K., Ritter, Y., and Şopu, D., \textit{Mech. Mater.}, \textbf{67}, 94–103 (2013)}
 \end{textblock*}

\end{frame}

\begin{frame}
 \frametitle{Aplicaciones}
 \centering
 Se trata de un material avanzado de ingenier\'ia
 
 \begin{textblock*}{12.6cm}(1.6cm,3cm)
  \begin{figure}
  \centering
  \begin{tabularx}{\textwidth}{cc}
  \only<1>{
    \subfloat[Joyer\'ia]{
      \includegraphics[height=3cm]{Cap_1/seamaster.png}}
    &
    \hspace{1cm}
    \subfloat[Deportes]{
      \includegraphics[height=3cm]{Cap_1/golf.png}}
    }
    \only<2>{
    \subfloat[MEMS]{
      \includegraphics[height=3cm]{Cap_1/MEMS_A.jpeg}}
    &
    \subfloat[MEMS]{
      \includegraphics[height=3cm]{Cap_1/MEMS_B.jpeg}}
    }
    \only<3>{
     \subfloat[Filo est\'andar]{
      \includegraphics[height=3cm]{Cap_1/blade.png}}
      &
      \hspace{1cm}
      \subfloat[Filo de BMG]{
	\includegraphics[height=3cm]{Cap_1/BMG-blade.png}}
    }
  \end{tabularx}
  \end{figure}
 \end{textblock*}
\end{frame}

\begin{frame}
 \frametitle{Objetivos}
 \vspace{0.5cm}
\begin{itemize}
 \item Investigar el comportamiento en r\'egimen elasto-pl\'astico en grandes deformaciones de un metal amorfo binario a diferentes temperaturas
 \begin{itemize}
  \item Tensión-deformación, parámetros constitutivos, dependencia con la temperatura.
 \end{itemize}
 \vspace{0.3cm}
 \item Investigar los efectos de cambios en la composici\'on sobre las propiedades mec\'anicas
 \begin{itemize}
  \item Generar muestras modificadas: porosidad variable e inclusiones cristalinas
 \end{itemize}
\end{itemize}
\end{frame}

\begin{frame}
 \frametitle{Metodología}
 \begin{figure}
  \centering
  \includegraphics[width=10cm]{Presentacion/multiescala.png}
 \end{figure}

\end{frame}

\begin{frame}
 \frametitle{Metodología}
 %\vspace{0.5cm}
 \begin{block}{Simulaciones Din\'amica Molecular (MD)}
  Resuelve ecuaciones de Newton para un sistema de N mol\'eculas que interact\'uan seg\'un una funci\'on potencial
 \end{block}
 \begin{textblock*}{12cm}(1.1cm,4.9cm)
  Proceso de simulación:
  \begin{enumerate}
   \item Pre-procesamiento de la muestra.
   \item Preparación del script de simulación para LAMMPS.
   \item Secuenciación de tareas.
   \item Ejecución de LAMMPS.
   \item Generación de gráficas ''publication-ready''.
   \item Análisis estadístico, de Voronoi, y visual.
  \end{enumerate}
 \end{textblock*}
\end{frame}

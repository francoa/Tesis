%%%%%%%%%%%%%%%%%%%%%%%%%%%%%%%%%%%%%%%%%%%%%%%%%%%%%
%		CONCLUSIONES
%%%%%%%%%%%%%%%%%%%%%%%%%%%%%%%%%%%%%%%%%%%%%%%%%%%%%

\section{Conclusiones}
\subsection{Conclusiones}

\begin{frame}
  \frametitle{CONCLUSIONES}
  \vspace{0.3cm}
 \begin{itemize} 
  \item Se aproxima el comportamiento en función de la temperatura con una función de decaimiento exponencial. No se encuentran trabajos en la literatura que realicen estas aproximaciones.
  \vspace{0.3cm}
  \item Se observa una marcada asimetría tracción - compresión. Pocos trabajos analizan distintos tipos de carga.
  \vspace{0.3cm}
  \item Se observan bandas de corte, las cuales son más evidentes con condiciones de frontera libre.
  \end{itemize}
\end{frame}

\begin{frame}
  \frametitle{CONCLUSIONES}
  \vspace{0.6cm}
 \begin{itemize} 
  \item Analizamos los efectos de la temperatura sobre nanopartículas embebidas, indicando estabilidad debajo de 400 K. A temperaturas más elevadas, se pierde la frontera nítida cristal/amorfo.
  \vspace{0.6cm}
  \item Las curvas tensión-deformación de la matriz con inclusión son muy similares al caso sin nanopartícula, pero existe un retardo en la nucleación de poros bajo tracción.
 \end{itemize}
\end{frame}

\begin{frame}
  \frametitle{CONCLUSIONES}
  \vspace{1cm}
 \begin{itemize}
  \item Muestra porosa bajo compresión: los poros concentran tensiones pero también retrasan la nucleación de SBs.
  \vspace{1cm}
  \item Muestra porosa bajo tracción: los poros no se cierran y facilitan el movimiento de átomos a su alrededor, impidiendo la formación de STZs y SBs.
 \end{itemize}
\end{frame}

\begin{frame}
 \frametitle{PRESENTACIONES EN CONGRESOS}
 \vspace{0.2cm}
 \begin{itemize}
  \item Trabajo completo publicado y presentado en el X Congreso Argentino de Mecánica Computacional (MECOM 2012)
  \vspace{0.2cm}
  \item Dos trabajos completos publicados y presentados en el I Congreso Panamericano de Mecánica Computacional (PANACM 2015)
 \end{itemize}
  \vspace{0.5cm}
  Además, se presentaron posters de estudiantes en el X Congreso Argentino de Mecánica Computacional (MECOM 2012) 
\end{frame}


\begin{frame}
 \frametitle{Trabajos futuros}
 \vspace{0.3cm}
 \begin{itemize}
  \item Generar y estudiar muestras a menores velocidades de enfriamiento y velocidad de deformación.
  \vspace{0.4cm}
  \item Estudiar muestras de mayor tamaño (mayor cantidad de nanopartículas y diferentes topologías de poros).
  \vspace{0.4cm}
  \item Llevar propiedades del estudio nano a códigos de elementos finitos para estudiar piezas de mayor tamaño.
  \vspace{0.4cm}
  \item Investigar el origen de la asimetría tracción - compresión.
 \end{itemize}
\end{frame}

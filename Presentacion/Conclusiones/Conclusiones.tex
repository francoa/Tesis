%%%%%%%%%%%%%%%%%%%%%%%%%%%%%%%%%%%%%%%%%%%%%%%%%%%%%
%		CONCLUSIONES
%%%%%%%%%%%%%%%%%%%%%%%%%%%%%%%%%%%%%%%%%%%%%%%%%%%%%

\section{Conclusiones}
\subsection{Conclusiones}

\begin{frame}
  \frametitle{CONCLUSIONES}
 \begin{itemize} 
  \item Se estudia una matriz de MG y se modifica su composición mediante simulaciones de MD.
  \vspace{0.2cm}
  \item Se caracteriza una muestra y se estudia el impacto de los cambios nombrados en la respuesta mecánica.
  \vspace{0.2cm}
  \item Para lograr lo anterior, se aprende a realizar simulaciones de MD.
  \vspace{0.2cm}
  \item Se presentan tres publicaciones: MECOM 2012 (1) y PANACM 2015 (2).
  \vspace{0.2cm}
  \item Se trabaja sobre la comunicación de resultados bajo diferentes requerimientos.
 \end{itemize}
\end{frame}

\begin{frame}
 \frametitle{Trabajos futuros}
 \vspace{0.5cm}
 \begin{itemize}
  \item Generar y estudiar muestras a menores velocidades de enfriamiento y velocidad de deformación.
  \vspace{0.5cm}
  \item Estudiar muestras de mayor tamaño (mayor cantidad de nanopartículas y diferentes topologías de poros).
  \vspace{0.5cm}
  \item Llevar propiedades del estudio nano a códigos de elementos finitos para estudiar piezas de mayor tamaño.
 \end{itemize}
\end{frame}

% Chapter Template

\chapter{INTRODUCCION} % Main chapter title

\label{C1} % Change X to a consecutive number; for referencing this chapter elsewhere, use \ref{ChapterX}

\lhead{Capítulo 1. \emph{INTRODUCCION}} % Change X to a consecutive number; this is for the header on each page - perhaps a shortened title

%----------------------------------------------------------------------------------------
%	SECTION 1
%----------------------------------------------------------------------------------------

\section{Descripción del problema y motivación. Tamaño (nano) de estudio}
\label{S1_1}

Los estados de agregación clásicos de la materia son tres: sólido, líquido y gaseoso. Los sólidos pueden ser clasificados en metales, cerámicos o polímeros. Esta clasificación se basa en la composicion química y la estructura atómica \citep{callister95}. Sin embargo, existe otra clasificación que, en aras de la claridad, suele separarse de los sólidos clásicos: los materiales avanzados. Un material avanzado puede ser un sólido clásico cuyas propiedades han sido mejoradas, pero también puede tratarse de nuevos materiales. En los últimos años, muchos de quiénes trabajan en el área de materiales se han concentrado en la búsqueda de nuevos materiales avanzados \citep{suryana11}. Su clasificación como material avanzado se basa en el uso que se le da: normalmente se trata de aplicaciones que requieren una gran tecnología y para las cuales los sólidos clásicos no son muy performantes.

En cuanto a la estructura de los sólidos clásicos, existen aquellos con estructura cristalina y otros con estructura amorfa. La estructura de un sólido se refiere al ordenamiento de los átomos en el material. Una estructura cristalina implicaría una disposición de los átomos con ordenamiento tridimensional periódico, formando cristales o granos bien definidos con cierta dirección. Los cristales o granos no son observables a simple vista en metales debido a su naturaleza opaca, pero son evidentes en los minerales \citep{smith96}. Los sólidos amorfos, por el contrario presentan un ordenamiento de corto alcance o ningún ordenamiento. De los sólidos clásicos, sólo los metales son cristalinos. Tanto los cerámicos como los polímeros presentan estructura amorfa.

Por otra parte, en cuanto a la composición química, la clasificación está basada en los elementos químicos que los componen. La tabla periódica de los elementos clasifica a los elementos químicos en elementos metálicos, no metálicos y elementos de transición. Un metal está compuesto de uno o más elementos metálicos (y eventualmente elementos no metálicos en muy baja concentración). Los cerámicos, por otro lado, están compuestos de una mezcla de elementos metálicos y no metálicos. 

Estos sólidos, debido a sus diferencias en estructura y composición química, exhiben propiedades mecánicas, térmicas, ópticas, etc. diferentes. Tanto metales como cerámicos suelen contar con una gran rigidez y resistencia. Los cerámicos suelen ser más duros, pero a la vez son más frágiles que los metales y son muy susceptibles a la fractura. En la figura \ref{C1:fg:propiedades} se definen algunas de las propiedades que normalmente se estudian en ciencia de los materiales.

- Imagen

De lo dicho anteriormente, se observa que tanto los metales como los cerámicos tienen ventajas y deficiencias uno con respecto al otro. Sería lógico entonces intentar combinar ambos tipos de sólido en un único material. Una de las formas de hacerlo es mediante los materiales compuestos. Los materiales compuestos representan una unión macroscópica (e incluso microscópica) de uno o más sólidos clásicos. Sin embargo, cada una de las partes que componen al material compuesto son un sólido clásico. Para lograr una combinación más fuerte es necesario reducir aún más la escala de trabajo, llegando a la escala nanométrica. Un nanómetro es equivalente a $10^{-9} m$. A modo comparativo, el tamaño promedio de un átomo suele tomarse como de 200 pm, lo cual implicaria que un nanómetro contendría 5 átomos.

El material denominado vidrio metálico (MG) logra el objetivo de combinar ambos tipos de sólido a escala nanométrica. Se le ha denominado \textit{vidrio} (un material cerámico) debido a su estructura amorfa y \textit{metálico} por tratarse de una aleación de elementos metálicos. Está basado en metales como cobre, níquel, hierro, oro, paladio, titanio, circonio, berilio y lantano; usualmente también se lo combina con bajas proporciones de no metales como boro, silicio, fósforo \citep{andrievski13}.

Teóricamente, todo líquido podría convertirse en vidrio a temperaturas suficientemente bajas siempre y cuando se evite el proceso de cristalización \citep{turnbull61}. El hecho de poder obtener una estructura diferente a la cristalina, implica que existe un equilibrio metaestable para una composición dada a cierta temperatura. Esto sucede cuando la energía libre es un mínimo, pero no un mínimo absoluto. Estas estructuras metaestables harian que la aleación pueda enfriarse de forma estable a baja temperatura, para la cual la viscosidad es tal, que la cristalización sería lo suficientemente lenta para permitir solidificar el líquido a una estructura amorfa.

A lo largo de los años se han ido perfeccionando varias técnicas que actúan sobre la composición, volumen y velocidad de enfriamiento a la que se somete el material (se detallarán algunas de estas técnicas en secciones subsiguientes), las cuales han permitido la síntesis de vidrios metálicos.

%----------------------------------------------------------------------------------------
%	SECTION 2
%----------------------------------------------------------------------------------------

\section{Antecedentes}
\label{S1_2}

%-----------------------------------
%	SUBSECTION
%-----------------------------------

\subsection{Experimentales}
\label{S1_2_1}

Los métodos tradicionales para la preparación de los vidrios metálicos son la solidificación del material fundido sobre una superficie en movimiento, la deposición en substratos fríos y la electrodeposición.

Según \cite{liebermann93} el primero en estudiar el enfriamiento rápido en aleaciones metálicas desde un punto de vista científico fue Pol Duwez. El propuso un método para obtener una lámina de metal amorfo a partir de un líquido que pasa a través de dos grandes ruedas de cobre a baja temperatura \citep{duwez60}. Al ser pequeña la relación entre la superficie de contacto del líquido y el volumen, éste se enfría de manera casi instantánea.

En un proceso similar (Torneado en estado de fusión o Melt Spinning en inglés), sólo se utiliza un sólo disco giratorio sobre el que se hace impactar un chorro líquido a alta presión sobre la cara exterior de la rueda, logrando velocidades de enfriamiento de hasta $10^{7}$ K/s.

Otra opción es atomizar un líquido y enfriar las pequeñas gotas mediante un gas. Se puede obtener así polvo de vidrio metálico (el cual es necesario sinterizar sin perder las propiedades inducidas por el enfriamiento), o hacer impactar estas gotas contra un molde para crear un sólido de forma directa. Existen métodos comerciales que logran una porosidad tan baja que el material así formado puede luego ser forjado.

Si bien el principio es el mismo, el resultado de estos procesos depende fuertemente de la aleación que se utilice, ya que la composición afecta en gran medida la capacidad de formación de vidrio del material. Es de interés entonces conseguir, además de altas velocidades de enfriamiento, composiciones que permitan producir vidrios a menores velocidades de enfriamiento.

Nombraremos algunos de los tantos trabajos que se realizaron sobre MGs. En el review de \cite{Axinte11} podemos encontrar numerosas aplicaciones de vidrios metálicos y referencias a trabajos pertinentes.

\cite{Zheng12} estudia el incremento de ductilidad en aleaciones cuando se reemplazan átomos por otros diferentes, modificando las propiedades de los enlaces atómicos. \cite{xiao12} observa la mecánica de deformación de MGs en nano-alambres bajo compresión, y cómo la velocidad de enfriamiento afecta la formación de bandas de corte. La tolerancia al daño de los MGs es analizada en \citep{Demetriou11}, en donde se indica que la relación dureza-resistencia de los mismos es comparable a la de los materiales más resistentes conocidos.

Un trabajo que estudia la dependencia de la resistencia y la ductilidad con el tamaño de muestra es \citep{Dongchan10}. En el mismo se realizan ensayos de tracción en nano-alambres de aleaciones basadas en Circonio de diferente diámetro junto con ciclos de carga y descarga.

El agregado de nano-precipitados es otro tema de estudio. \cite{kuo14} analiza cómo el agregado de precipitados con estructura $B2$ modifica la plasticidad inducida por deformación. Un estudio sobre la formación de esos precipitados mediante recocido se encuentra en \citep{wei14}. En este último, primero se obtiene una matriz amorfa de MG y luego se la recoce alrededor de $T_{g}$ para lograr cristalizar estructuras $B2$ en una proporción cercana a la deseada.

%-----------------------------------
%	SUBSECTION
%-----------------------------------

\subsection{Estudios Teóricos}
\label{S1_2_2}

Si bien se comenzó por generar aleaciones en la práctica, es muy importante poder modelar de alguna manera, teóricamente, el comportamiento de los materiales bajo diferentes condiciones. El efecto del ordenamiento atómico, de su composición, los mecanismos de deformación y fractura. Comprender desde una base teórica estos puntos hace posible predecir el comportamiento de estructuras que todavía no se consiguen a gran escala pero que son prometedoras para un futuro cercano.

+Voronoi
+Limiting Strength
+Failure criterion
+

%----------------------------------------------------------------------------------------
%	SECTION 3
%----------------------------------------------------------------------------------------

\section{Alcances}
\label{S1_3}


%----------------------------------------------------------------------------------------
%	SECTION 4
%----------------------------------------------------------------------------------------

\section{Objetivos}
\label{S1_4}

%----------------------------------------------------------------------------------------
%	SECTION 5
%----------------------------------------------------------------------------------------

\section{Contenidos}
\label{S1_5}


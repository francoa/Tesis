% Chapter Template

\chapter{CONCLUSIONES} % Main chapter title

\label{C6} % Change X to a consecutive number; for referencing this chapter elsewhere, use \ref{ChapterX}

\lhead{Capítulo 6. \emph{CONCLUSIONES}} % Change X to a consecutive number; this is for the header on each page - perhaps a shortened title

%----------------------------------------------------------------------------------------
%	SECTION 1
%----------------------------------------------------------------------------------------

\section{Generales}

\cambioGrande{MODIFICACION GENERAL}

En el presente trabajo se realizaron simulaciones atomísticas del comportamiento mecánico de vidrios metálicos volumétricos (BMGs) bajo diferentes modos de carga y con diferentes modificaciones a su composición haciendo uso de simulaciones de dinámica molecular (MD). En el \cref{C2} se presentan algunos de los códigos desarrollados en este trabajo para el análisis de los resultados obtenidos en las simulaciones. En el \cref{C3} se estudia una matriz amorfa de cobre-circonio (Cu$_{46}$Zr$_{54}$) bajo tensiones de tracción y compresión y se observan los efectos de la temperatura en los parámetros constitutivos así como en la respuesta mecánica. Luego, en el \cref{C4}, se estudia la estabilidad de nanopartículas cristalinas en la matriz amorfa previamente caracterizada y se compara el comportamiento mecánico con los resultados del \cref{C3}. Por último, en el \cref{C5}, se analiza la influencia de la fracción de volumen sólido sobre el comportamiento mecánico del material estudiado, utilizando muestras con distintos niveles de porosidad.

Como podemos observar en las curvas de tensión-deformación del \cref{C3}, el comportamiento tanto en tracción como en compresión de la muestra original presenta una variación en función de la temperatura. Los valores del módulo de Young y tensión de Von Mises máxima se reducen entre un 30 y 40\% al aumentar de bajas temperaturas a temperaturas cercanas a la transición vítrea. Se obtuvieron buenos ajustes de las curvas al utilizar una fórmula de decremento exponencial con la temperatura, típico de procesos activados térmicamente. Se observa un claro apartamiento del caso a 900 K del resto de las simulaciones, donde el incremento de la temperatura produce un decremento considerable del módulo elástico y de la tensión de Von Mises. Esto es de esperar al ser una temperatura muy superior a la temperatura de transición vítrea (696 K). Cabe destacar que en todos los casos se encuentra una asimetría en tracción y compresión, y los valores se encuentran más próximos sólo en el caso del módulo de Young. Las tensiones máximas son mucho mayores en compresión.

En lo que respecta al comienzo de la plasticidad, se han observado indicios de banda de corte (SB), en el caso de la muestra bajo condiciones de borde periódicas, aunque no tan definidas. Esto es de esperarse por dos motivos: en primer lugar, la elevada velocidad de enfriamiento de la muestra y en segundo lugar porque no existen concentradores de esfuerzos al realizar simulaciones con condiciones de borde periódicas. Observamos un cambio importante en el comportamiento plástico al simular la misma muestra bajo condiciones de borde libres, puesto que se ven SBs bien definidas. Esto ocurre porque las superficies libres poseen una mayor energía libre y son zonas donde pueden nuclearse STZs y SBs con mayor facilidad.

Al estudiar los efectos de la temperatura sobre nanopartículas embebidas, los ajustes de difusividad en función de la temperatura pudieron realizarse para el caso de partícula Cu-FCC con $T \geq 500 K$ donde existen desplazamientos atómicos lo suficientemente importantes. Esto puede interpretarse como el hecho de que la partícula es estable a temperaturas menores a 500 K, es decir, no se disuelve en la matriz. Sin embargo, a temperaturas más elevadas los átomos difunden y se pierde la estructura cristalina y con esto la interfase definida entre material cristalino y amorfo. Observando las curvas de desplazamientos para el caso de la partícula CuZr-B2 se observa una pendiente cercana a cero en todos los casos, lo que dificulta el ajuste de difusividad. Si bien esto puede interpretarse como la estabilidad de la partícula, hay que tener en cuenta que los desplazamientos iniciales en iguales condiciones son mucho mayores para la partícula CuZr-B2 que para la partícula Cu-FCC.

A la hora de realizar cargas sobre la matriz con inclusiones cristalinas se aplicaron condiciones de frontera periódicas en tres dimensiones, y la ausencia de superficies libres deja sólo a la nanopartícula como probable concentrador de esfuerzos para promover la nucleación de STZs y así desencadenar las bandas de corte. Cabe destacar que esta concentración de esfuerzo no es suficiente para desencadenar SBs en ningún caso (como sí podemos observar en otros trabajos \citep{albe13,brink15,adibi13,adibi14}) y si bien en todos los casos de tracción se nuclean poros (lo cual no es sorprendente luego de haberse encontrado con el mismo fenómeno al estudiar la matriz amorfa), los mismos nuclean en zonas alejadas de la nanopartícula.

Si bien las curvas de esfuerzo-deformación son claramente similares al caso sin nanopartícula, encontramos que a excepción de un caso (partícula CuZr-B2 a 200 K) se produce un retardo en la nucleación de un poro al traccionar la muestra. En el caso excepcional, la nucleación se produce antes que en la muestra original. El análisis de Voronoi no muestra diferencias significativas entre las muestras con y sin inclusión de nanopartícula. 

Los resultados de deformar la matriz original con el agregado de porosidad fueron comparables a aquellos encontrados en la literatura \citep{yuan14} para la compresión de muestras porosas de monocristales de cobre. Esto puede ser considerado como una validación del proceso de sinterizado utilizado para la preparación de las muestras y explicado en la \sref{S5_3}.

Se estudiaron muestras con distintas fracciones de volumen sólido (SVF) bajo carga uniaxial compresiva y de tracción. Con carga compresiva, los poros facilitan la plasticidad actuando como concentradores de tensiones, pero también retrasan la formación de zonas de transformación de tensión cortante (STZs) y su posible unión en una banda de corte (SB), para el material lejano a los poros. Los resultados también exhiben un endurecimiento de la muestra al cerrarse los poros, de forma similar a lo que ocurre en el caso no poroso. Con carga de tracción y deformación puramente uniaxial, los poros no cierran y concentran flujo plástico alrededor de ellos, a su vez que también impiden la formación de STZs y bandas de corte.

El proceso de sinterizado utilizado posee ciertos parámetros que pueden ser modificados a fin de poder observar su influencia en las propiedades mecánicas de la muestra porosa. Al reducir la velocidad de cambio de temperatura durante el sinterizado, las redes de poros interconectadas son reemplazadas por un único poro de forma esférica. Este cambio de forma, sin embargo, no implica un cambio en el volumen total de los poros, el cual se mantiene similar para las dos velocidades estudiadas. Las nuevas muestras obtenidas de esta manera presentan una tensión máxima aumentada y mayor módulo de elasticidad. Por otro lado, se demostró que la ubicación de los poros no influye en las propiedades mecánicas de la muestra, siempre y cuando la ubicación sea aleatoria.

%----------------------------------------------------------------------------------------
%	SECTION 2
%----------------------------------------------------------------------------------------

\section{Trabajos Futuros}

Existen dos factores que influyen considerablemente a la hora de evidenciar la plasticidad heterogénea de los BMGs: la velocidad de enfriamiento y la velocidad de deformación. Es por esto que en estudios futuros y a fines comparativos, es esencial generar muestras a velocidades de enfriamiento menores y realizar experimentos simulados a una velocidad de deformación uno o dos ordenes de magnitud menor. Hay que considerar que esto último representa un costo computacional mucho mayor.

Otro factor que limita las posibilidades de análisis es el tamaño de la muestra. Contar con muestras de mayor tamaño permitiría estudiar el efecto de un  número mayor de nanopartículas embebidas que el caso puntual estudiado en el \cref{C4}, así como lograr muestras porosas con diferentes topologías y una distribución más uniforme de los poros.

El análisis de Voronoi, lejos de ser sencillo y directo, requiere un estudio muy detallado para obtener conclusiones concretas y comparables con otros estudios. Esto es algo a realizar en futuras simulaciones.

Una vez caracterizado el material a nivel nano en el \cref{C3}, el paso siguiente es ingresar estas propiedades mecánicas en códigos de elementos finitos para realizar simulaciones en piezas de mayores dimensiones para observar la respuesta del material a nivel macro.

\section{Publicaciones y Presentaciones en Congresos}

En el contexto de este Proyecto Final de Estudios y los proyectos ''06/B235 Estudio de Materiales Amorfos`` y ''B008 Estudio de Posibles Mejoras de las Propiedades Mecánicas de Vidrios Metálicos`` financiados por la Secretaría de Ciencia, Técnica y Postgrado (SeCTyP) de la Universidad Nacional de Cuyo, hemos participado y publicado en conferencias y congresos científicos.

En el año 2012, se presentó un paper titulado ''ATOMISTIC SIMULATIONS OF AMORPHOUS METALS IN THE ELASTOPLASTIC REGIME``, durante el X Congreso Argentino en Mecánica Computacional (MECOM 2012) en Salta, Argentina. En dicho congreso también se presentaron dos pósteres de estudiantes.

En el año 2015, se presentaron dos papers titulados ''ATOMISTIC STUDY OF THE MECHANICAL PROPERTIES OF A SINTERED BULK METALLIC GLASS (NANOGLASS)`` y ''MECHANICAL PROPERTIES OF A CU$_{46}$ ZR$_{54}$ BULK METALLIC GLASS WITH EMBEDDED CRYSTALLINE NANO PARTICLES``, durante el 1$^{er}$ Congreso Pan-Americano en Mecánica Computacional (PANACM 2015) y el XI Congreso Argentino en Mecánica Computacional (MECOM 2015) en Buenos Aires, Argentina.

Los papers mencionados se presentaron en carácter de trabajo completo publicado y pueden encontrarse anexados al final del trabajo.

En el año 2015 se presentó un poster en el VIII Encuentro de Investigadores y Docentes de Ingeniería (EnIDI 2015), titulado ''Estudio numérico de vidrios metálicos con inclusiones de nanopartículas y nanoporos''.


%% CAP 3

%Atomistic simulations of bulk metallic glasses (BMGs) mechanical behavior under tension and compression were performed using molecular dynamics (MD) simulations. 

%The increase of sample temperature produces a considerable decrease of the samples elastic modulus. The same applies to maximum von Mises stress. It is observed that the elastic modules are practically the same under tension or compression at different temperatures, but the maximum stress in compression is much higher. The behavior with temperature can be adjusted reasonably well with an exponential decay with temperature, typical of thermal activated phenomena.

%No shear bands are observed, which is to be expected given that our glass was generated with very high quenching rates. Since no shear bands are observed in our simulations, the identification of plasticity is complex. Surely there are shear areas, "shear transformation zones" (STZ), composed of a few atoms that experience high shear stresses. The identification of these areas requires a very detailed observation of the sample, involving much longer simulations than those used here. An alternative to study plasticity is the examination of Voronoi polyhedra, which can help to identify these areas. Such studies are in progress.

%In the future, using more powerful computational resources than available for this work, we plan to create samples with quenching rates orders of magnitude slower, with the aim to observe the possible formation of shear bands.

%A detailed understanding of the influence of temperature, quenching rates, etc., in the mechanical properties of metallic glasses will allow obtaining necessary properties for their application in new technologies, including applications under extreme conditions, such as aerospace missions or materials in nuclear reactors. Studies like the one presented here will contribute to this understanding and accelerate novel material development. 

%% CAP 4

%Estudiamos un BMG con una nanopartícula cristalina como inclusión. Consideramos un vidrio CuZr, y una nanopartícula de Cu pura con un radio de 2 nm. Ésto implica una fracción en volumen que varía desde 1.15\% a 10 K hasta 1.12\% a 800 K como resultado del aumento del volumen inicial de la muestra con la temperatura. Una situación similar fue explorada recientemente por Albe et al. \citep{albe13}. Aquí, nos centramos en los efectos de la temperatura, e inicialmente estudiamos la estabilidad debajo de los 400 K, indicando que la nanopartícula es bastante estable a esas temperaturas. A temperaturas mayores, la difusividad en sólo algunos ns trae consigo la pérdida de una interfaz nítida entre la nanopartícula y la matriz.

% En nuestras simulaciones, se aplicaron condiciones de frontera periódicas en tres dimensiones, y la ausencia de superficies libres deja sólo a la nanopartícula como probable concentrador de esfuerzo para promover la nucleación de STZs, y así desencadenar las bandas de corte, en la interfaz entre la matriz y la nanopartícula. Sin embargo, éste no fue el caso. Las curvas de esfuerzo-deformación son claramente similares al caso sin nanopartícula, a excepción de un retardo en la nucleación de un poro bajo tracción para la muestra con una nanopartícula.

% El análisis de Voronoi no muestra diferencias significativas entre las muestras con y sin inclusión de nanopartícula. Un estudio futuro y más detallado es requerido para diferentes modos de carga y temperaturas. Estudios futuros también podrían repetir estos experimentos con inclusiones de CuZr con una estructura cristalina B2, como podemos encontrar en algunas experiencias \citep{wei14,kuo14}.

%% CAP 5

%Se realizaron simulaciones de Dinámica Molecular (MD) en una muestra porosa del vidrio metálico Cu$_{46}$ Zr$_{54}$, aplicando esfuerzos de compresión y tracción. Los resultados bajo deformación fueron comparables a aquellos encontrados en la literatura \citep{yuan14} para la compresión de muestras porosas de monocristales de cobre. Esto puede ser considerado como una validación del proceso de sinterizado utilizado para la preparación de las muestras.

%Con carga compresiva, los poros facilitan la plasticidad actuando como concentradores de tensiones, pero también retrasan la formación de zonas de transformación de tensión cortante (STZs) y su posible unión en una banda de corte (SB), para el material lejano a los poros. Los resultados también exhiben un endurecimiento de la muestra al cerrarse los poros, similarmente a lo que ocurre en el caso no poroso.

%Con carga de tracción y deformación puramente uniaxial, los poros no cierran y concentran flujo plástico alrededor de ellos, a su vez que también impiden la formación de STZs y bandas de corte.

% Estudios futuros incluirán un análisis de Voronoi profundizado y la simulación de muestras más grandes con topologías de porosidad diferentes.
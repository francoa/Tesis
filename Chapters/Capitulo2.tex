% Chapter Template

\chapter{ASPECTOS TEORICOS BASICOS} % Main chapter title

\label{C2} % Change X to a consecutive number; for referencing this chapter elsewhere, use \ref{ChapterX}

\lhead{Capítulo 2. \emph{ASPECTOS TEORICOS BASICOS}} % Change X to a consecutive number; this is for the header on each page - perhaps a shortened title

%----------------------------------------------------------------------------------------
%	SECTION 1
%----------------------------------------------------------------------------------------

\section{Simulación, Teoría y Experimentos}
\label{S2_1}

Las simulaciones en computadora han abierto la posibilidad a estudiar sistemas, y verificar modelos bajo condiciones que son imposibles (o muy difíciles) de conseguir experimentalmente. También han permitido incrementar la complejidad de los modelos usados y realizar comparaciones directas con fenómenos naturales, una vez que el modelo ha sido validado.

Con la llegada de computadoras más potentes, aparece una nueva opción a considerar entre teoría y experimentos: los experimentos simulados. El modelo sobre el cual se realiza esta simulación sigue siendo teórico, pero todos los cálculos son llevadoa cabo por la computadora. Es muy importante por supuesto, definir las condiciones de simulación de tal manera que represente una situación fisicamente posible, y no obtener resultados absurdos.

Tanto el modelo como las condiciones de simulación y los elementos que componen el experimento simulado se encuentran en el marco de algún paquete de software que mediante algoritmos específicos resuelven el problema y nos entregan ciertos resultados. En este caso particular se hace uso de simulaciones de Dinámica Molecular (apropiadas para estudiar fenómenos en escala nano) mediante el paquete LAMMPS [REF], que es de código abierto y gratuito.

%----------------------------------------------------------------------------------------
%	SECTION 2
%----------------------------------------------------------------------------------------

\section{Introducción a MD}
\label{S2_2}

%----------------------------------------------------------------------------------------
%	SECTION 3
%----------------------------------------------------------------------------------------

\section{Potenciales, Termostatos, etc etc}
\label{S2_3}


%----------------------------------------------------------------------------------------
%	SECTION 4
%----------------------------------------------------------------------------------------

\section{Modelo mecánico}
\label{S2_4}


%----------------------------------------------------------------------------------------
%	SECTION 5
%----------------------------------------------------------------------------------------

\section{Modelo numérico}
\label{S2_5}


%-----------------------------------
%	SUBSECTION
%-----------------------------------

\subsection{EDO que resulve LAMMPS}
\label{S2_5_1}
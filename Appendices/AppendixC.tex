% Appendix Template

\chapter{SCRIPTS DE CONSOLA Y AWK} % Main appendix title

\label{AC} % Change X to a consecutive letter; for referencing this appendix elsewhere, use \ref{AppendixX}

\lhead{Anexo C. \emph{SCRIPTS DE CONSOLA Y AWK}} % Change X to a consecutive letter; this is for the header on each page - perhaps a shortened title

Una herramienta muy potente de los sistemas operativos tipo UNIX son los scripts de consola o \textit{shell scripts}. En el presente
capítulo se resumirán las capacidades y ventajas que posee esta herramienta, y además se introducirá una herramienta apropiada para
el procesamiento rápido de datos y la modificación de archivos llamada \textit{awk}.

\section{SHELL SCRIPTS}

Los scripts de consola son programas que pueden ser ejecutados en terminales o consolas de comandos de sistemas operativos tipo UNIX. La programación de scripts de consola no presenta grandes ventajas, especialmente si se lo compara con algunos de los lenguajes de programación más actuales. Sin embargo, los comandos de consola UNIX pueden ser directamente utilizados en un shell script, lo cual permite la automatización de tareas que de otro modo serían repetitivas y engorrosas. En otras palabras, el aprendizaje de la programación de scripts de consola va de la mano con el aprendizaje del uso del sistema operativo UNIX. 

\subsection{PROGRAMACION BASICA}

Para comenzar, es preciso crear un archivo en el cual escribiremos el código. Dicho archivo no requiere ninguna extensión en particular,
únicamente debe tener permisos de ejecución. Los siguientes tres comandos sirven a este fin:
\begin{lstlisting}
 echo e > shellScriptFile
 ls -la | grep shellScriptFile
 chmod +x shellScriptFile
\end{lstlisting}

El comando \textit{echo} envía a la salida estándar lo que se le pasa por argumento. El argumento, en el caso del ejemplo, es la letra $e$.
Al usar el símbolo $>$ se está redirigiendo la salida estándar a un archivo nuevo (que será creado automáticamente) llamado \textit{shellScriptFile}. Dicho archivo contendrá únicamente la letra $e$.

El comando \textit{ls} devuelve una lista de las carpetas y archivos presentes. La opción \textit{-la} permite obtener mayor información, como por ejemplo los permisos de lectura, escritura y ejecución para cada grupo de usuarios. El símbolo $|$ es lo que se denomina \textit{pipe}, y sirve para utilizar la salida de un comando como entrada de otro. En este caso, la salida de \textit{ls} será enviada al comando \textit{grep}, que es un comando de búsqueda. De esta manera, el comando completo \textit{ls -la $|$ grep shellScriptFile} devolverá información completa para todos los archivos o directorios que se llamen o contengan dentro de su nombre el texto shellScriptFile.

Por último, el comando \textit{chmod} nos permite cambiar los permisos a un archivo. Al escribir \textit{+x} como argumento estamos indicando que queremos que el archivo \textit{shellScriptFile} pueda ser ejecutado por todos los usuarios existentes en el sistema.

Estos cuatro comandos (echo, ls, grep y chmod), son comandos de consola UNIX. Por ende pueden ser utilizados en un shell script. Tomemos como ejemplo el siguiente script:
\begin{lstlisting}
 #!/bin/bash
 echo $1
\end{lstlisting}

Para indicar al sistema que se trata de un shell script, debemos incluir el identificador \textit{\#!}, seguido por la dirección del interpretador de comandos. \textit{Bash y sh} son las opciones más usuales. 

La línea de código \textit{echo \$1} indicaría que se imprimirá en consola \textit{``\$1''}. Sin embargo cuando se ejecuta el script (comando de ejecución :\textit{./shellScriptFile}), lo único que resulta es una línea en blanco.  Esto es porque el símbolo \$ es un símbolo reservado. Este símbolo se utiliza para indicar que queremos el valor de la variable. Por ejemplo \textit{echo var} imprimirá en la pantalla el texto \textit{var}, mientras que \textit{echo \$var} imprimirá en la pantalla el valor almacenado en la variable llamada \textit{var}.

¿Qué significa entonces la variable 1? El símbolo \$ seguido por un número indica una variable que almacena los argumentos del script. Es decir, si se ejecuta el script mediante \textit{./shellScriptFile textoRandom} se leerá en pantalla la palabra \textit{textoRandom}. Un script con el siguiente llamado : \textit{./script abc 123 xyz} almacenará ``abc'' en la variable \$1, 123 en la variable \$2, ``xyz'' en la variable \$3 y así sucesivamente.

Otros comandos útiles son \textit{cd} para desplazarse de una carpeta a otra del sistema de archivos, \textit{pwd} para mostrarnos
en qué carpeta estamos ubicados actualmente, \textit{mkdir} para crear una carpeta, \textit{cp, mv y rm} para copiar, mover o eliminar
archivos y/o carpetas. Se recomienda profundizar sobre éste tema e incrementar el conocimiento de comandos de Linux. Muy útil suele ser revisar las entradas del manual, utilizando el comando \textit{man}, seguido por el nombre del comando del cual queremos obtener información.

\subsection{EXPRESIONES CONDICIONALES Y LOOPS}

Las expresiones condicionales y loops o bucles forman parte de prácticamente todos los lenguajes de programación. Los lenguajes de shell script no son la excepción. Suele resultar más engorroso utilizarlos que en otros lenguajes, ya que las expresiones de comparación son diferentes y un espacio mal ubicado puede hacer fallar el programa.
\begin{lstlisting}
#!/bin/bash
let cond=1
let cond2=2
if [ $1 -le $cond ];
then
        echo "menor a 1"
elif [ $1 -gt $cond2 ];
then
        echo "mayor a 2"
else
        echo "entre 1 y 2"
fi
\end{lstlisting}
\begin{lstlisting}
#!/bin/bash
i=$1
while [ $i -le $2 ];
do
        echo $i
        let i=i+1
done
\end{lstlisting}

La ubicación de los espacios parece caprichosa, pero es necesario que se dejen los espacios como se presentan aquí. Por ejemplo, el comando
\textit{let} permite asignar un valor numérico a una variable. Si en vez de \textit{let cond=1} se hubiese escrito \textit{let cond = 1} (dejando espacios a ambos lados del signo igual) el interpretador hubiese arrojado error. Otro ejemplo se da en las expresiones de decisión condicional (\textit{if then else}) y  en las expresiones de terminación de bucles: si se hubiese escrito \textit{if [\$1 -le \$cond]} el interpretador hubiese arrojado error por no dejar espacios entre variables y corchetes.

Las expresiones de comparación no se basan en símbolos, como en la mayoría de los lenguajes de programación. En la \tref{AC:tbl:comp}
se muestran las expresiones.
\begin{table}[htp]
\begin{center}
\begin{tabular}{*{2}{c}}
\hline
-eq&	igual (equal)\\ \hline
-ne&	desigual (not equal to)\\ \hline
-lt&	menor a (less than)\\ \hline
-le&	menor o igual a (less than or equal to)\\ \hline
-gt&	mayor a (greater than)\\ \hline
-ge&	mayor o igual a (greater than or equal to)\\ \hline
\end{tabular}
\end{center}
\caption[Expresiones de comparación en shell script]{Expresiones de comparación en shell script.}
\label{AC:tbl:comp}
\end{table}

\section{AWK}

Awk es una excelente herramienta para procesar tablas de datos, y una vez que se aprende el lenguaje que utiliza awk resulta más sencillo de aplicar y más rápida que otras herramientas.

La estructura básica de un comando awk desde consola es la siguiente:
\begin{lstlisting}
awk 'BEGIN { print "START" }
	   { print $1      }
     END   { print "STOP"  }' file
\end{lstlisting}

Donde se aplica el comando al archivo \textit{file}. En la porción de \textit{BEGIN} deben incluirse todas las inicializaciones de variables y configuraciones necesarias (por ejemplo, la definición de los separadores de campos \textit{FS}). En la porción de \textit{END} normalmente se incluyen comandos de escritura. En la porción intermedia se realiza todo el procesamiento de los datos.
\begin{lstlisting}
 awk 'BEGIN {}
      {
      counter=0
	for(i=1;i<=10;i++) {
	    if(i!=NR) {
		counter+=i
	    }
	}
	print counter;
      } 
      {
      counter=0
	for(i=1;i<=10;i++) {
	    if(i!=NR) {
		counter+=$i
	    }
	}
	print counter;
      }
      END {}' archivoDePrueba
\end{lstlisting}

En el código presentado, es posible apreciar que es un lenguaje similar a C. Los bucles y expresiones condicionales se realizan de igual manera que en C. La gran diferencia reside en que, por defecto, awk ejecuta las porciones de código entre corchetes (aquellas que no están precedidas ni por BEGIN ni por END) tantas veces como número de registros (filas) haya en el archivo de entrada. Es decir que con un archivo que contenga 10 filas, ejecutará 10 veces las porciones de código intermedias. Esto permite iterar fácilmente a través de todo el archivo y acceder de forma sencilla al valor de las columnas o campos. A éstos valores se accede haciendo uso del símbolo \$. \$1 dará acceso al valor del primer campo o columna.

Además del signo \$, hay palabras que están reservadas. Presentamos algunas en la \tref{AC:tbl:awkReserved}.
\begin{table}[htp]
\begin{center}
\begin{tabular}{*{2}{c}}
\hline
FS&	Separador de campos o columnas (Field Separator)\\ \hline
NF&	Número de campos (Number of Fields)\\ \hline
RS&	Separador de registros o filas (Record Separator)\\ \hline
NR&	Número de registro actual (Number of input record)\\ \hline
OFS&	Separador de campos en la salida (Output field separator)\\ \hline
ORS&	Separador de registros en la salida (Output record separator)\\ \hline
\end{tabular}
\end{center}
\caption[Expresiones reservadas en awk]{Expresiones reservadas en awk.}
\label{AC:tbl:awkReserved}
\end{table}
% Appendix Template

\chapter{SCRIPTS DE CONSOLA Y AWK} % Main appendix title

\label{AC} % Change X to a consecutive letter; for referencing this appendix elsewhere, use \ref{AppendixX}

\lhead{Anexo C. \emph{SCRIPTS DE CONSOLA Y AWK}} % Change X to a consecutive letter; this is for the header on each page - perhaps a shortened title

Una herramienta muy potente de los sistemas operativos tipo UNIX son los scripts de consola o \textit{shell scripts}. En el presente
capítulo se resumirán las capacidades y ventajas que posee esta herramienta, y además se introducirá una herramienta apropiada para
el procesamiento rápido de datos y la modificación de archivos llamada \textit{awk}.

\section{SHELL SCRIPTS}

Los scripts de consola son programas que pueden ser ejecutados en terminales o consolas de comandos de sistemas operativos tipo UNIX.
La programación de scripts de consola no presenta grandes ventajas, especialmente si se lo compara con algunos de los lenguajes de programación
más actuales. Sin embargo, los comandos de consola UNIX pueden ser directamente utilizados en un shell script, lo cual permite la automatización
de tareas que de otro modo serían repetitivas y engorrosas. Básicamente, el aprendizaje de la programación de scripts de consola va de la mano
con el aprendizaje del uso del sistema operativo UNIX a través de la terminal. ??? REVISAR TODO EL PARRAFO

\subsection{PROGRAMACION BASICA}

Para comenzar, es preciso crear un archivo en el cual escribiremos el código. Dicho archivo no requiere ninguna extensión en particular,
únicamente debe tener permisos de ejecución. Los siguientes tres comandos sirven a este fin:
\begin{lstlisting}
 echo e > shellScriptFile
 ls -la | grep shellScriptFile
 chmod +x shellScriptFile
\end{lstlisting}

El comando \textit{echo} envía a la salida estándar lo que se le pasa por argumento. El argumento, en el caso de ejemplo, es la letra $e$.
Al usar el símbolo $>$ se está redirigiendo la salida estándar a un archivo nuevo (que será creado automáticamente) llamado \textit{shellScriptFile}.
Dicho archivo contendrá únicamente la letra $e$.

El comando \textit{ls} devuelve una lista de las carpetas y archivos presentes. La opción \textit{-la} permite obtener mayor información, como por
ejemplo los permisos de lectura, escritura y ejecución para cada grupo de usuarios. El símbolo $|$ es lo que se denomina \textit{pipe}, y sirve
para utilizar la salida de un comando como entrada de otro. En este caso, la salida de \textit{ls} será enviada al comando \textit{grep}, que es un
comando de búsqueda. De esta manera, el comando completo \textit{ls -la $|$ grep shellScriptFile} devolverá información completa para todos los
archivos o directorios que se llamen o contengan dentro de su nombre el texto shellScriptFile.

Por último, el comando \textit{chmod} nos permite cambiar los permisos a un archivo. Al escribir \textit{+x} como argumento estamos indicando
que queremos que el archivo \textit{shellScriptFile} pueda ser ejecutado por todos los usuarios existentes en el sistema.

Estos cuatro comandos (echo, ls, grep y chmod), son comandos de consola UNIX. Por ende pueden ser utilizados en un shell script. Tomemos como
ejemplo el siguiente script:
\begin{lstlisting}
 #!/bin/bash
 echo $1
\end{lstlisting}

Para indicar al sistema que se trata de un shell script, debemos incluir el identificador \textit{\#!}, seguido por la dirección del interpretador
de comandos. \textit{Bash y sh} son las opciones más comúnes. 

La línea de código \textit{echo \$1} indicaría que se imprimirá en consola \textit{\$1}.
Sin embargo cuando se ejecuta el script (comando de ejecución :\textit{./shellScriptFile}), lo único que resulta es una línea en blanco. 
Esto es porque el símbolo \$ es
un símbolo reservado. Con él estamos indicando que queremos el valor de la variable. Por ejemplo \textit{echo var} imprimirá en la pantalla
el texto \textit{var}, mientras que \textit{echo \$var} imprimirá en la pantalla el valor almacenado en la variable llamada \textit{var}.

¿Qué significa entonces la variable 1? El símbolo \$ seguido por un número indica una variable que almacena los argumentos del script. Es decir, si
se ejecuta el script mediante \textit{./shellScriptFile textoRandom} se leerá en pantalla la palabra \textit{textoRandom}. Un script con el
siguiente llamado : \textit{./script abc 123 xyz} almacenará ``abc'' en la variable \$1, 123 en la variable \$2, ``xyz'' en la variable \$3 y
así sucesivamente.

Otros comandos útiles son \textit{cd} para desplazarse de una carpeta a otra del sistema de archivos, \textit{pwd} para mostrarnos
en qué carpeta estamos ubicados actualmente, \textit{mkdir} para crear una carpeta, \textit{cp, mv y rm} para copiar, mover o eliminar
archivos y/o carpetas. Se recomienda profundizar sobre éste tema e incrementar el conocimiento de comandos de linux. Muy útil suele ser revisar
las entradas del manual, utilizando el comando \textit{man}, seguido por el nombre del comando del cual queremos obtener información.

\subsection{EXPRESIONES CONDICIONALES Y LOOPS}




\section{AWK}